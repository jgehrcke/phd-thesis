\chapter{Development and validation of a molecular dynamics-based docking method (DMD)}
\section{Rational}
A particularly challenging class of ligands for which docking performance is
especially limited by the above-mentioned issues are glycosaminoglycans
(GAGs), which are periodic negatively charged linear polysaccharides mainly located
in the extracellular matrix. Through interaction with their protein targets
they participate
in a number of key processes involved in cell regeneration and proliferation,
angiogenesis, metastasis and lipid metabolism \cite{hynes_extracellular_2009,
macri_growth_2007, barbero_chembiochem_2013}. Due to the occurrence of
numerous sulfate and carboxyl groups, GAGs have a high
charge density, rendering long-range Coulomb interactions to
be crucial for binding to proteins \cite{mulloy_specificity_2005}. The
significance of electrostatic interactions underlines the importance
of taking explicit solvent molecules as binding mediators into account when GAGs
are used as ligands in molecular docking \cite{samsonov_docking_2011}. Moreover,
the orientation and conformation of long side chains of charged protein residues
may be greatly influenced by interaction with a GAG ligand. In consequence, flexible
treatment of the receptor during GAG-docking is of special relevance.
In addition, a similar spatial distribution of functional groups in GAGs independent of the
reducing/non-reducing end orientation \cite{forster_computational_2006} as well as their
high flexibility \cite{bitomsky_docking_1999} substantially contribute to
the challenges in protein-GAG docking. Overall, the prediction of protein-GAG complexes
via molecular docking comprises a good example of some of the current limitations
in the field of classical molecular docking.

Classical docking approaches are generally optimized for having relatively modest computational
requirements and therefore enable the quick
investigation of single complexes as well as the execution of large-scale
studies involving a large number of different complexes. Taking into account
ligand and receptor flexibility as well as treating solvent explicitly clearly
increases the computational complexity of docking approaches. However, in view of the
ever-increasing computing power, a new generation of docking methods should
evolve, which though being computationally more demanding, aims to deal
with the above-mentioned challenges.

Molecular dynamics (MD) techniques are established for rigorous studies of intermolecular interactions\cite{karplus_molecular_2005}.
Beyond that, MD methods have already been used in docking approaches
for overcoming the challenges of both ligand and receptor flexibility\cite{chaudhuri_application_2012, antes_dynadock_2010}.
Furthermore, standard MD methods
allow for the inclusion of explicit solvent via well-established water models.
The application of MD techniques is limited by high computational cost, which previously
has hindered their usage in high-throughput approaches for drug discovery.
However, nowadays, advanced computational resources are
available and specialized hardware (such as graphics processing units, GPUs) can
be used to dramatically increase the performance of MD simulations, making the
establishment of MD in the field of docking gradually feasible.


We propose Dynamic Molecular Docking (DMD), a combination of established MD-based
methods specifically designed for tackling the above-mentioned challenges in protein-GAG
docking.
DMD is a targeted MD-based approach where the ligand, which is initially
placed at a distance from the receptor so that their interaction is negligible,
is slowly pulled towards a receptor target region by applying a
time-dependent distance restraint. During this process, both receptor
and ligand are treated as entirely flexible in explicit solvent. The time-dependent distance
restraint applied in DMD is usually used in steered molecular dynamics
simulations, which are performed for studying the energetics of
processes that generally happen on timescales too large for being treated by
classical MD simulations \cite{xiong_free_2006}, such as protein folding, ligand
unbinding and large-scale conformational alteration. In DMD, once the ligand reaches
the receptor, the distance restraint is switched off and a long free MD simulation is carried out,
allowing for the mutual adjustment of receptor residues and GAG as well as for
extensive GAG-internal degree of freedom sampling.
The obtained trajectory data are then used for extracting a
docking solution (the coordinates) and for its characterization
(binding energy estimate and other quantities).
Binding pose search as
well as binding pose scoring are consistent in terms of using the same potential
as given by the MD force field and the parameterization of the molecular system.

In this validation study, we applied DMD to a set of reference systems and compared its
performance to a classical docking approach, AutoDock 3 (AD3), which has successfully been applied to protein-GAG systems before {\cite{japan_docking_ad3_clustering, samsonov_docking_2011,
pichert_characterization_2012}}. The results obtained for our test data set comprised of
five protein-GAG complexes, one protein-peptide and one protein-small molecule complex show
that DMD has high predictive significance
while performing best when complex formation is driven by strong electrostatic
interaction. Detailed analysis of the electrostatic potential of the
corresponding binding sites, of the spatial distribution of docking solutions, and
per-residue binding free energy decomposition support the promising potential of
the DMD approach when used for docking in highly flexible and electrostatics-driven
systems.
\section{General method description}
\subsection{Dynamic Molecular Docking protocol}
For each complex in the TDS, DMD was carried out in the following way.
Starting with randomly oriented ligand molecules placed at a distance from the receptor
(ligand re-oriented models, LROMs), 100 independent DMD runs were performed.
The first step of a DMD run is a targeted molecular dynamics (tMD) simulation in which
the ligand is pulled towards a pre-defined target region on the receptor via a time-dependent decrease of the
distance $d(t)$ between one central atom in the ligand and one core atom in the
protein receptor. Among DMD run repetitions, most of the ligand trajectories
lie within a certain \enquote{entry lane} which is focused on a
point near the receptor surface, the focus point $\bm{F}$, and therefore
defining the target region (Figure 1).
All final tMD states have the central ligand atom positioned on the surface of a sphere defined by
the protein core atom (the center of the sphere) and the final distance $D$ of the
tMD pulling process (the radius of the sphere).
Based on the final state of each
tMD simulation, the second step of a DMD run relaxes the system via a free MD
 simulation. Geometrical definitions, system preparation, and details about tMD
and free MD parameterization as well as subsequent trajectory data analysis
methods are provided in the following sections.


\subsubsection{Preparation of ligand-reoriented models (LROMs)}
An LROM contains the receptor as well as the ligand placed in a distal, re-oriented position.
Per TDS complex, ten LROMs have been created,
differing only in ligand orientation around its central atom. For a given complex, LROM creation requires the
selection of a \textit{central ligand atom}, definition of a \textit{focus point} $\bm{F}$ near the
surface of the receptor within the anticipated binding region, definition of a
\textit{ligand displacement length} $s$ and selection of a \textit{core atom} at point $\bm{C}$ within the
receptor. The distance between focus point and core atom defines the final
distance $D$ of the tMD pulling process, i.e.\  $d(T) \equiv D \equiv  \lVert \bm{F}-\bm{C} \rVert$.
Initially, at time $t_0$, the ligand is placed distal from the receptor with its central atom lying
on the axis defined by $\bm{F}-\bm{C}$ (Figure 1b). The starting coordinate for the central
ligand atom is defined as

\begin{equation}
\bm{L}(t_0) = \bm{F} + s \frac{\bm{F}-\bm{C}}{D}.
\end{equation}

For each TDS complex, ten LROMs were prepared as follows. First, the structure of the biological unit of
the protein receptor was
taken from the corresponding experimental data source. The coordinates of a
central atom in the ligand as found in the experimentally determined structure
were used as focus point. The core atom in the receptor was selected fulfilling
three criteria: \textit{i)} it is a backbone atom within a helix or beta sheet in the
protein core, \textit{ii)} the line connecting core atom and $\bm{F}$ (defining the orientation of
the ``entry lane'' is roughly perpendicular to the surface comprising the
anticipated binding region, and \textit{iii)} the
surface of the sphere around the core atom with radius $D$ has significant overlap
with the molecular receptor surface in the receptor target region. Distal
ligand placement was followed by ten random ligand rotations uniformly
distributed in 3D space with $\bm{L}(t_0)$ being the rotation center. From each
rotational ligand state one LROM was built.


\subsubsection{Molecular dynamics protocol}
All molecular dynamics simulations were set up, performed, and analyzed
using Amber 11 and AmberTools 12 \cite{case_amber_11}. The FF99SB force field was used for
parameterization of the peptidic parts in the TDS complexes. Hyaluronan, chondroitin
sulfate and heparin monosaccharide force field parameters were created based on
GLYCAM 06 version g \cite{kirschner_glycam06:_2008} and sulfate partial
charges obtained by RESP fitting calculations at the level of 6-31(d)G for
methylsulfate.
The trypsin inhibitor was parameterized using the GAFF force field via
AmberTools' antechamber program. All systems were solvated in a box of TIP3P
water with a minimum of $8\,\angstrom$ distance between solute and box boundaries. The
systems were neutralized by adding Na$^{+}$ or Cl$^{-}$ counterions. During MD, lengths of
bonds including hydrogen atoms were constrained by SHAKE. The time integration
step was set to $2\,\mathrm{fs}$. Non-bonded interactions were cut off for distances larger
than $8\,\angstrom$. The Particle Mesh Ewald method was used for treating long-range
electrostatic interactions.


Each simulated system went through minimization, heat up, equilibration, and
production steps. During the first stage of minimization, only the solvent was
relaxed. During the second stage, the entire system was minimized without
restraints. System heat up to $300\,\mathrm{K}$ was performed within $20\,\mathrm{ps}$ in the canonical
ensemble (NVT) using the Langevin thermostate and periodic boundary conditions.
Subsequently, $500\,\mathrm{ps}$ of MD in the isothermal-isobaric ensemble (NPT) with
Langevin thermostate and Berendsen barostat under periodic boundary conditions
were carried out for system equilibration. The following production stage of
duration $T$ was performed in the NVT ensemble with the Berendsen thermostat and
periodic boundary conditions.

For the complexes involving FGF2 and heparin, weak torsional restraints were applied in order to keep the
pyranose rings of IdoA(2S) in the $^{1}C_4$ conformation. This conformation has been shown to be one
of the two predominantly populated ones{\cite{almond_jacs_2010}}
and was observed experimentally in the structure of the FGF2-HP complex in our test data set (PDB ID 1BFB).
The applied restraints enable to define and control the specific conformation of each IdoA(2S) ring
throughout the entire DMD study, since its natural ring conformer population is not properly
reproduced by GLYCAM 06{\cite{gandhi_idoa2s_2010}}.





{\sffamily \small Targeted and free molecular dynamics.}
The LROMs were prepared for MD and time-evolved following the general MD
protocol as described above. During the tMD production stage, core
atom and ligand center atom were exposed to an additional time-dependent
harmonic potential
\begin{equation}
U(t) = \frac{1}{2} k \left( d(t)-d(t_0) + vt   \right)^2
\end{equation}
with force constant $k=200\,\mathrm{kcal\,mol^{-1}\,A^{-2}}$, pulling velocity
$ v = s/T$ and
\begin{equation}
d(t) = \lVert \bm{L}(t)-\bm{C}(t) \rVert.
\end{equation}
This potential enforces the distance between the selected ligand center atom and
the protein core atom to linearly decrease with time $t$ in the interval from $D+s$ to $D$
(Figure 1c).

For all tMD simulations, we used a pulling velocity of $s=30\,\angstrom$ per $T=4\,\mathrm{ns}$.
The tMD pulling process was repeated ten times with different
random seeds for each of the ten LROMs per complex, yielding 100 independent tMD
simulations for each TDS complex. Based on the final state of receptor and
ligand in each tMD trajectory, an MD simulation with
$T=10\,\mathrm{ns}$ without restraints was performed according to the protocol described
above, hereafter referred to as free MD.
\section{Implementation for this project}
\lipsum[1-5]
\section{Validation study}
\lipsum[1-5]
