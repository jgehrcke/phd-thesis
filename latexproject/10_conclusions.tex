\chapter{Conclusions}

In this PhD project it was my goal to reveal structural and molecular details
about IL-10-GAG interaction with theoretical and computational means, and with
the help of experiments performed by collaborators.

Application of a novel method for protein-GAG binding region analysis to IL-10
lead to the reliable prediction of a IL-10-GAG binding region centered on the
residues R102, R104, R106, and R107 of the human IL-10 monomer. Based on this
finding, the application of another novel method, Dynamic Molecular Docking
(DMD), in combination with a clustering method optimized for protein-GAG
systems, was able to derive further molecular detail about IL-10-GAG
interaction: it was found that R107 significantly stands out compared to all
other amino acid residues in IL-10 and supposedly plays a particularly important
role in IL-10-GAG recognition. Since this conclusion has been generated
\textit{in silico}, it is important to note that it is based on different
reliable types of data, including ensemble- and time-averaged quantities
obtained from molecular dynamics data collected on the microsecond time scale.
Additionally, it was revealed that K119 potentially plays an important role, as
it might cooperate with R107 in trapping a GAG from two sides simultaneously.
Application of DMD to the IL-10-GAG system led to the observation of two
principally different GAG binding orientations, which could both be valid in the
natural IL-10-GAG system.

Generally, it was found that GAGs do not bind to IL-10 with a binding affinity
as large as for instance in the FGF2-HP system. Still, especially with the help
of NMR data, it was revealed that IL-10-HP interaction occurs in a systematic
fashion, with a measurable impact on the HP backbone structure, as observed via
molecular dynamics-based HP structure modeling based on nuclear Overhauser
effect data obtained by our collarboration partners at the University Leipzig.

ooperative binding behavior
- binding in that region might affect R2 binding according to the model
    presented in ...






    - Insights for IL-10-GAG system
        - binding site
        - binding behavior (heparin, N sulfate, conformation, important residue)
        - binding implications


In the course of the
project, three methods were developed for the \textit{in silico} investigation
of protein-GAG complexes and subsequently applied to the IL-10-GAG system.


    - Methodological development
        - Binding site prediction
        - Clustering
        - DMD
        - Experimental setup: HPC, ZIH and group-internal


    no real binding specificity clarified, e.g. effect of N sulfation not clarified. this requires further in silico sampling techniques, and for the sake of creating reliable results, a clear and certain knowledge about the binding region / site (this is something for outlook). Also, further experimental data needs to be obtained, in vitro experiments, SPR with mutants, etc.


