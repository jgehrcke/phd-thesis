\chapter{Conclusions and outlook}

In this PhD project it was my goal to reveal structural and molecular details
about IL-10-GAG interaction with theoretical and computational means, and in
combination with experimentally obtained data provided by collaborators.
Furthermore, since the \textit{in silico} methods for protein-GAG investigation
are still in their infancy, this thesis project included the development and
improvement of methodological approaches.

Application of a novel method for protein-GAG binding region analysis to IL-10
lead to the prediction of an IL-10-GAG binding region centered on the residues
R102, R104, R106, and R107 of the human IL-10 sequence. Based on this finding,
the application of another novel method, Dynamic Molecular Docking (DMD), in
combination with a clustering method optimized for protein-GAG systems, was able
to derive further molecular details about IL-10-GAG interaction: it was found
that R107 significantly stands out compared to other amino acid residues in
IL-10 and supposedly plays a particularly important role in IL-10-GAG
recognition. Since this conclusion has been generated \textit{in silico}, it is
important to note that it is based on different reliable types of data,
including ensemble- and time-averaged quantities obtained from molecular
dynamics data collected on the microsecond time scale. Additionally, it was
revealed that residue K119 potentially plays an important role, as it might
cooperate with R107 in trapping a GAG from two sides simultaneously. Application
of DMD to the IL-10-GAG system led to the observation of two principally
different GAG binding orientations, which could both be valid in the natural
IL-10-GAG system. Generally, it was found that GAGs do not bind to IL-10 with
such a high affinity as in most previously investigated protein-GAG systems
(e.g. FGF2-HP). Still, especially with the help of NMR data, it was revealed
that IL-10-HP interaction occurs in a non-random fashion, with a measurable
impact on the HP backbone structure, as observed via molecular dynamics-based HP
structure modeling incorporating nuclear Overhauser effect data obtained by our
collaboration partners at the Universität Leipzig. Considering the
aforementioned findings, the obvious experimental tasks are now the
investigation of the role of R107 for GAG-binding affinity in NMR, SPR or ITC
experiments, and the role of R107 for the impact of GAG-binding on IL-10 biology
via e.g.\ \textit{in vitro} cell proliferation experiments, whereas all of these
studies do require the creation of corresponding IL-10 mutants.

\nomenclature{SPR}{surface plasmon resonance}
\nomenclature{ITC}{isothermal titration calorimetry}

Behavioral differences among different GAG types were observed in this project,
in the sense that all binding effects were more pronounced with an increasing
number of sulfate groups per disaccharide unit. This was shown especially in
terms of the hydrogen bonding data and the single-residue energy decomposition
data as obtained via DMD. Certain observations from NMR experiments and also
from \textit{in vitro} experiments suggested that particular GAG sulfate groups
might have a special role in IL-10-GAG interaction \cite{salek_ardakani_2000,
kuenze_gehrcke_2014}. In this PhD project, this was so far not confirmable.
While such a discovery is possible in the future using DMD, more experimental
data is required for conducting corresponding \textit{in silico} studies. In
particular, experimental confirmation of the IL-10-GAG binding site would be
very helpful for the success of further \textit{in silico} investigations.

The data obtained about the IL-10-GAG system so far allows for hypothesizing two
scenarios in which GAG-binding to IL-10 affects its biological function. In one
scenario, long GAG chains impair the diffusion of the IL-10 cytokine and
therefore might take part in a complex cytokine concentration regulation system.
In the second scenario, GAG-binding to IL-10 sterically interferes with the
binding of IL-10R2 to IL-10, which could directly inhibit the activation of the
IL-10-triggered cell-internal signaling cascade. Clearly, more reliable data
about the structure of the ternary IL-10 signaling complex than current models
provide are required to obtain further insights in this regard. Currently, the
cooperative binding behavior in which one GAG molecule interacts with two
separate binding regions simultaneously (discussed in
\cref{nmr:further_insights}) fits best the idea of polymeric GAGs catching IL-10
in the groove of the V-shape and impairing its diffusion.

During this thesis project, valuable structural insights for increasing the
understanding about the IL-10-GAG system and its molecular mechanisms were
derived, and also the yield in novel \textit{in silico} methods for the generic
investigation of protein-GAG systems can be considered rather successful. The
binding region prediction method presented in \cref{chapter:bspred} provides a
valuable bare-bones tool for future investigators of protein-GAG systems, with
clear advantages over existing methods. The molecular structure clustering
method described in \cref{chapter:clustering} is based on a solid concept,
involves components heavily optimized for protein-GAG systems, and serves as a
useful and sometimes \textit{essential} data post-processing technique. Dynamic
Molecular Docking (DMD) can be considered the most valuable core result of this
thesis project: it is an advanced \textit{in silico} method for the
investigation of protein-GAG systems, with one clear goal: yielding reliable
data based on large-scale molecular dynamics simulations. DMD could largely
facilitate and improve the investigation of protein-GAG systems in the future,
and renders a perfect triad together with the newly developed protein-GAG
binding region prediction and docking solution clustering.
