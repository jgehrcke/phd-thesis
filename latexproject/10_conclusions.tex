\chapter{Conclusions}


- cooperative binding behavior
- binding in that region might affect R2 binding according to the model
    presented in ...




    - Methodological development
        - Binding site prediction
        - Clustering
        - DMD
        - Experimental setup: HPC, ZIH and group-internal
    - Insights for IL-10-GAG system
        - binding site
        - binding behavior (heparin, N sulfate, conformation, important residue)
        - binding implications


    no real binding specificity clarified, e.g. effect of N sulfation not clarified. this requires further in silico sampling techniques, and for the sake of creating reliable results, a clear and certain knowledge about the binding region / site (this is something for outlook)


    Furthermore, K119 — which
was left undiscussed in chapter 2 — potentially plays an important
role, as it might cooperate with R107 in trapping a GAG from two
sides simultaneously. This structural model of IL-10-GAG interaction
is supported by clustering data that led to the observation of two
principally different GAG binding poses in the region of interest, both
heavily involving R107