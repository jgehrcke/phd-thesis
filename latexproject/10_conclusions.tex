\chapter{Conclusions and outlook}

In this PhD project it was my goal to reveal structural and molecular details
about IL-10-GAG interaction with theoretical and computational means, and with
the help of experiments performed by collaborators.

Application of a novel method for protein-GAG binding region analysis to IL-10
lead to the reliable prediction of a IL-10-GAG binding region centered on the
residues R102, R104, R106, and R107 of the human IL-10 monomer. Based on this
finding, the application of another novel method, Dynamic Molecular Docking
(DMD), in combination with a clustering method optimized for protein-GAG
systems, was able to derive further molecular detail about IL-10-GAG
interaction: it was found that R107 significantly stands out compared to all
other amino acid residues in IL-10 and supposedly plays a particularly important
role in IL-10-GAG recognition. Since this conclusion has been generated
\textit{in silico}, it is important to note that it is based on different
reliable types of data, including ensemble- and time-averaged quantities
obtained from molecular dynamics data collected on the microsecond time scale.
Additionally, it was revealed that K119 potentially plays an important role, as
it might cooperate with R107 in trapping a GAG from two sides simultaneously.
Application of DMD to the IL-10-GAG system led to the observation of two
principally different GAG binding orientations, which could both be valid in the
natural IL-10-GAG system.

Generally, it was found that GAGs do not bind to IL-10 with a binding affinity
as large as for instance in the FGF2-HP system. Still, especially with the help
of NMR data, it was revealed that IL-10-HP interaction occurs in a systematic
fashion, with a measurable impact on the HP backbone structure, as observed via
molecular dynamics-based HP structure modeling based on nuclear Overhauser
effect data obtained by our collaboration partners at the University Leipzig.

\nomenclature{SPR}{surface plasmon resonance}
\nomenclature{ITC}{isothermal titration calorimetry}

Behavioral differences among different GAG types were observed, in the sense
that all effects were more pronounced with an increasing number of sulfate
groups per disaccharide unit. This was shown especially in terms of the hydrogen
bonding data and the single-residue energy decomposition data as obtained via
DMD. However, the impact of certain sulfate groups in the IL-10-GAG system was
so far not clarified using pure \textit{in silico} investigation, so that
corresponding observations made in NMR experiments and also published by
Salek-Ardakini et al. \cite{salek_ardakani_2000} could unfortunately not be
verified or structurally explained so far. While this type of analysis is
possible using \textit{in silico} sampling techniques, a clear and absolutely
certain knowledge about the GAG binding site should be available from
experiments --- for the sake of creating \textit{reliable} results. Hence, more
experimental data is required with respect to the IL-10-GAG system for the
success of further \textit{in silico} investigations. After what has been found
out so far, the obvious experimental tasks are now the investigation of the role
of R107 for GAG-binding affinity in NMR, SPR or ITC experiments, and the role of
R107 for the impact of GAG-binding on IL-10 biology via e.g.\ \textit{in vitro}
cell proliferation experiments, whereas all of these studies do require the
creation of corresponding IL-10 mutants.

The kind of data obtained about the IL-10-GAG system so far allows two realistic
scenarios in which GAG-binding to IL-10 affects the biological function of
IL-10. In one scenario, long GAG chains impair the diffusion of the IL-10
cytokine and therefore may take part in a complex cytokine concentration
regulation system. In the second scenario, GAG-binding to IL-10 sterically
interferes with the binding of IL-10R2 to IL-10, which could directly inhibit
the activation of the IL-10-triggered cell-internal signaling cascade. Clearly,
more reliable data about the structure of the ternary IL-10 signaling complex
than current models provide are required to provide further insights in this
regard. Currently, the cooperative binding behavior in which one GAG molecule
interacts with two separate binding regions simultaneously (observed via NMR
experiments \cite{kuenze_gehrcke_2014}) fits best the idea of polymeric GAGs
catching IL-10 in the groove of the V-shape.

While the insights into the molecular mechanism of the IL-10-GAG system obtained
in the course of this project might be slightly disappointing, the yield in
novel \textit{in silico} methods for the generic investigation of protein-GAG
systems can be considered rather successful. The binding region prediction
method presented in \cref{chapter:bspred} provides a valuable bare-bones tool
for future investigators of protein-GAG systems. The docking solution
clustering method described in \cref{chapter:clustering}

project, three methods were developed for the \textit{in silico} investigation
of protein-GAG complexes and subsequently applied to the IL-10-GAG system.


    - Methodological development
        - Binding site prediction
        - Clustering
        - DMD
        - Experimental setup: HPC, ZIH and group-internal





