\chapter{A simple method for predicting protein-GAG binding sites}

\subsubsection{Electrostatic potential analysis in protein-GAG systems}

\hl{Generalize this section, keep details for DMD chapter. At the moment most of
this is simply copied from the DMD manuscript.}

The electrostatic potential often dominates protein-GAG interaction
\cite{gandhi_structure_2008}. In this section, we discuss the electrostatic
properties of the receptors in the TDS with the goal to determine how these
properties could assist defining a receptor target region for DMD and also to be
able to relate docking performance to electrostatic characteristics of the
receptor.


Poisson-Boltzmann-based numeric calculations of the electrostatic potential of
molecules are most error-prone near the dielectric boundary, i.e.\ on the
molecular surface. We therefore did not simply map the potential on the
molecular surface but analyzed the topology of the potential with an isosurface
representation while varying the isosurface value. This allows for an
understanding  of the distribution of the potential in space and how strongly it
would affect a ligand. \cref{fig:bspred:sdf1_estatic} shows an isosurface of the
Coulomb potential of SDF-1. \hl{Supplementary Figure 1} shows analogous
isosurface representations of the electrostatic potential for the other TDS
complexes. For this type of visualization, the isovalue selection was done
individually for each receptor in the TDS such that only a small part of the
isosurface is protruding into space further than the molecular surface of the
receptor.

\begin{figure}
\centering
\includegraphics[width=0.9\textwidth]{gfx/bspred/figure_3_sdf1_isopot_8_5_view1_rotated_pub_001.pdf}
\caption[]{
Isosurface representation of SDF-1's electrostatic potential with the isovalue $\Phi = 8.5\,\mathrm{kcal\,mol^{-1}\,e^{-1}}$ (blue). The molecular surface of the SDF-1 dimer is
shown in grey, the heparin ligand as determined experimentally (PDB ID 2NWG) is shown as sticks
with C atoms in orange.
}
\label{fig:bspred:sdf1_estatic}
\end{figure}

Regarding the SDF-1-HP complex, our electrostatic potential evaluation procedure
unambiguously identifies the GAG binding site as determined experimentally. With
$8.5\,\mathrm{kcal\,mol^{-1}\,e^{-1}}$, the isovalue chosen is the largest among
the TDS complexes, indicating that SDF-1 has the strongest electrostatic
attraction to its ligand. In case of FGF2-HP (\hl{Supplementary Figure 1}), the
binding site is also unambiguously defined by the electrostatic potential. For
CD44-HA, the net electrostatic interaction between both molecules is slightly
repulsive. There is no obvious relation between the electrostatic properties of
the receptor and the binding site location. CathK and CathKmut exhibit
electrostatic attraction for negatively charged molecules in the experimentally
observed ligand binding site as well as in an adjacent region. We observe that
electrostatic potential analysis provides a clear idea whether GAG-binding to a
given receptor is mainly driven by Coulomb interaction. If a protein is known to
bind GAGs, and the electrostatic potential topology is as unambiguous as in case
of SDF-1 or FGF2, a GAG binding site prediction based on the presented procedure
is reliable. Visualization of the electrostatic potential can also be helpful to
\textit{a priori} exclude regions of the receptor surface when repulsive to
negatively charged ligands. Furthermore, this analysis shows that knowledge
about the electrostatic potential distribution in space can be used to choose a
reasonable ligand \enquote{entry lane} orientation for the tMD pulling process.


The SH3-p41 complex is dominated by non-polar interactions, rendering the
Coulomb potential analysis inconclusive. Recognition of the Trypsin inhibitor in
its  binding pocket is affected by polar interactions. The electrostatic
potential, however, does not display clear characteristics to predict a binding
region.

