\chapter{Protein-GAG binding site prediction via electrostatic potential analysis}

In this chapter we present a simple and yet efficient method for predicting
where on a protein a GAG would most likely bind, suggesting which amino acid
residues are likely to be involved in the interaction and which are not. The
method is based on numerical simulation of the electrostatic (i.e.\ Coulomb)
potential of the protein in water solution, and manual evaluation of the
topology of the Coulomb potential in space. Subsequently, we apply that method
to IL-10 and report our findings.

\section{Motivation}

Classical molecular docking approaches are used for creating binding predictions
for a given receptor-ligand system. Primarily, the correctness of a prediction
depends on the quality of the underlying physical or phenomenological model, and
whether the system under investigation is within the scope of validity of the
model. Since most molecular docking methods are trained for analyzing a limited
Cartesian volume in a \textit{local} search, application to larger search spaces
dramatically lowers the confidence in the resulting prediction. As a result,
common molecular docking methods should be applied locally, and therefore
require \textit{a priori} knowledge about where on the receptor the ligand most
likely binds. The decision where to center the local search on should be based
on reliable data, otherwise the docking study might be pointless. In the best
case, knowledge about the binding site (the region on the receptor that the
ligand binds to) comes from experimental data, e.g.\ from mutagenesis or NMR
studies. However, in many cases no experimental data is available that provides
clues about the binding site, as is unfortunately true for the IL-10-GAG system.
In such situations, it makes sense to consider \textit{in silico} methods for
binding site prediction.

With respect to protein-GAG systems, there is only few published work on this
topic. In \cite{hp_binding_sites_mulloy_2006}, Forster and Mulloy used AutoDock
version 2.4 \cite{autodock24} for globally docking rigid heparin molecules to
antithrombin and FGF-2, and found that their docking procedure gave good
agreement with the corresponding crystal structures regarding the overall
position of the heparin binding sites. The authors state that the method may be
used as \enquote{hypothesis generation tool} rather than for providing
\enquote{details of interactions between specific atoms}. In another work
\cite{gandhi_bmp_heparin_binding_sites_2012}, the authors used PatchDock
\cite{patchdock_2002} for globally docking rigid heparin oligosaccharides to
BMP-2 and BMP-14 and state that \enquote{while there has been no validation of
the accuracy of the PatchDock scoring function for heparin interactions, these
docking results suggest the presence of two GAG binding sites in BMP-2}.

The above-cited studies do by no means provide a complete picture, but obviously
some of the established docking methods seem to be able to reflect certain
properties of protein-GAG systems, although none of these docking methods has
been specifically optimized for GAG ligands, let alone for \textit{global}
docking. Strictly spoken, when applied to protein-GAG systems, these docking
methods are applied outside of their scope of validity. We postulate that the
reason for the surprising success in named studies is that protein-GAG complexes
are strongly driven by Coulomb interaction and that mentioned docking methods
incorporate Coulomb interaction with a significant weight in their scoring
functions.

In fact, heparin has the highest negative charge density among biological
macromolecules \cite{capila_linhardt_hep_prot_2002}. Standard literature on
protein-GAG systems such as
\cite{essentials_glycobiology_gags_2009} stresses the role of charge-charge
interaction in protein-GAG systems in general. Many publications focusing on
individual protein-GAG systems identify charge complementary as one of the
driving mechanisms of the interaction between GAG and protein
\cite{gandhi_bmp_heparin_binding_sites_2012,faham_heparin_1996,%
pichert_characterization_2012,rogers_gag_prot_prot_2011}.

%dominates all other contributions and therefore a heterogeneously charged
%protein is able to \enquote{catch} GAG ligands.

We can conclude that the importance of Coulomb interaction is a unique feature
of protein-GAG systems. Notably, compared to other molecular interactions, the
Coulomb interaction is a long range interaction and therefore dominates all
other contributions for larger distances. As of these considerations, it seems
to be reasonable to perform protein-GAG binding site prediction purely based on
the strength and topology of the electrostatic potential of the protein. Whether
or not such an approach proves to be useful when applied to a set of reference
systems in any case further enlightens the role of Coulomb interaction in
protein-GAG systems.

%None of the authors provides arguments why the docking method of choice should
%be valid for protein-GAG systems.
%
% In that case,
%the docking study might be able to provide insights about the receptor- ligand
%interaction on an atomic level, i.e.\ with higher resolution than the
% and
%also on the sampling performance of the method. We can safely conclude that with
%increasing search space, the validity of any prediction

%The larger the cartesian search space, the
%
%The validity of binding predictions from classical molecular docking approaches
%increases with


\section{Method}

\subsection{Coulomb potential simulation}

\subsection{Coulomb potential evaluation}

\section{Application to reference systems}


\begin{figure}
\centering
\includegraphics[width=0.9\textwidth]{gfx/bspred/fgf2_coulomb_isosurfaces_different_values_03_ds.pdf}
\caption[]{
Isosurface representation of FGF2's Coulomb potential (blue), shown for multiple
isovalues isovalues $\Phi$. The molecular surface of FGF2 dimer is shown in
gray, its heparin ligand pose as determined experimentally is shown as sticks
with C atoms in orange (structure taken from PDB ID 1BFB). }
\label{fig:bspred:fgf2_multi_iso}
\end{figure}



\subsection{Results and discussion}

TODO: argue, why application of complex docking methods only allowed us to
speculate why they are sometimes succcessfull (see motivation part), but now,
in a controlled environment with a very simple model (only coulomb), we can
directly conclude that coulomb interaction IS the dominating interaction that
alone determines binding site for many systems.


\section{Application to IL-10}


\begin{figure}
\centering
\includegraphics[width=0.9\textwidth]{gfx/bspred/SI_figure_IL-10_coulomb_isosurface_1_9kcalmol.png}
\caption[]{

Isosurface representation of IL-10's Coulomb potential with the isovalue $\Phi =
\SI{1.9}{\kilo\calory\per\mole\per\elementarycharge}$ (blue). The molecular
surface of the IL-10 dimer is shown in gray, the molecular surface of arginines
102, 104, 106, 107 is shown in yellow (structure taken from PDB ID 2ILK). This
representation allows to see where the Coulomb potential (for a given isovalue)
protrudes into space further than the molecular surface, and has been shown to
provide useful evidence about where GAGs bind to a protein \hl{(REF)}. Hence,
IL-10-GAG interaction most likely takes place within the two symmetrically
arranged regions indicated here.

}
\label{fig:bspred:il10_estatic_pred}
\end{figure}




\subsubsection{Electrostatic potential analysis in protein-GAG systems}

\hl{Generalize this section, keep details for DMD chapter. At the moment most of
this is simply copied from the DMD manuscript.}

The electrostatic potential often dominates protein-GAG interaction
\cite{gandhi_structure_2008}. In this section, we discuss the electrostatic
properties of the receptors in the TDS with the goal to determine how these
properties could assist defining a receptor target region for DMD and also to be
able to relate docking performance to electrostatic characteristics of the
receptor.


Poisson-Boltzmann-based numeric calculations of the electrostatic potential of
molecules are most error-prone near the dielectric boundary, i.e.\ on the
molecular surface. We therefore did not simply map the potential on the
molecular surface but analyzed the topology of the potential with an isosurface
representation while varying the isosurface value. This allows for an
understanding  of the distribution of the potential in space and how strongly it
would affect a ligand. \cref{fig:bspred:sdf1_estatic} shows an isosurface of the
Coulomb potential of SDF-1. \cref{fig:bspred:various_estatic} shows analogous
isosurface representations of the electrostatic potential for the other TDS
complexes. For this type of visualization, the isovalue selection was done
individually for each receptor in the TDS such that only a small part of the
isosurface is protruding into space further than the molecular surface of the
receptor.



\begin{figure}
\centering
\includegraphics[width=0.9\textwidth]{gfx/bspred/sdf1_isopot_8_5_view1_rotated_jcc_pub_001.jpg}
\caption[]{
Isosurface representation of SDF-1's electrostatic potential with the isovalue
$\Phi = 8.5\,\mathrm{kcal\,mol^{-1}\,e^{-1}}$ (blue). The molecular surface of
the SDF-1 dimer is shown in grey, the heparin ligand as determined
experimentally (PDB ID 2NWG) is shown as sticks with C atoms in orange.
}
\label{fig:bspred:sdf1_estatic}
\end{figure}

Regarding the SDF-1-HP complex, our electrostatic potential evaluation procedure
unambiguously identifies the GAG binding site as determined experimentally. With
$8.5\,\mathrm{kcal\,mol^{-1}\,e^{-1}}$, the isovalue chosen is the largest among
the TDS complexes, indicating that SDF-1 has the strongest electrostatic
attraction to its ligand. In case of FGF2-HP (\cref{fig:bspred:various_estatic}), the
binding site is also unambiguously defined by the electrostatic potential. For
CD44-HA, the net electrostatic interaction between both molecules is slightly
repulsive. There is no obvious relation between the electrostatic properties of
the receptor and the binding site location. CathK and CathKmut exhibit
electrostatic attraction for negatively charged molecules in the experimentally
observed ligand binding site as well as in an adjacent region. We observe that
electrostatic potential analysis provides a clear idea whether GAG-binding to a
given receptor is mainly driven by Coulomb interaction. If a protein is known to
bind GAGs, and the electrostatic potential topology is as unambiguous as in case
of SDF-1 or FGF2, a GAG binding site prediction based on the presented procedure
is reliable. Visualization of the electrostatic potential can also be helpful to
\textit{a priori} exclude regions of the receptor surface when repulsive to
negatively charged ligands. Furthermore, this analysis shows that knowledge
about the electrostatic potential distribution in space can be used to choose a
reasonable ligand \enquote{entry lane} orientation for the tMD pulling process.


\begin{figure}
\centering
\includegraphics[width=0.9\textwidth]{gfx/bspred/suppl_figure_estatic_distributions_004_1200.png}
\caption[]{
Visualization of the electrostatic potential of the receptors from the TDS
(molecular surfaces in gray) with their ligands as experimentally determined
(shown as sticks with C atoms in orange). Blue/red: Isosurface representation of
the receptor's electrostatic potential with isovalue Ф (attractive/repulsive for
negatively charged molecules, respectively).
}
\label{fig:bspred:various_estatic}
\end{figure}


\hl{Create more convincing Figure than bspred:variousestatic.
Include more interesting complexes, remove uninteresting ones.}

The SH3-p41 complex is dominated by non-polar interactions, rendering the
Coulomb potential analysis inconclusive. Recognition of the Trypsin inhibitor in
its  binding pocket is affected by polar interactions. The electrostatic
potential, however, does not display clear characteristics to predict a binding
region.



\hl{INCORPORATE:} The method surely is more useful than the popular method of
heparin binding site consensus sequence search (cardin, weintraub) (hileman,
linhardt), which has already been applied to interleukin-10 before. Its about
structure, not sequence, that is also why Forster and Mulloy state that
\enquote{though some 'consensus sequences' for heparin binding have been
identified, they are neither necessary nor sufficient to define a heparin
binding site} \cite{forster_computational_2006}.
