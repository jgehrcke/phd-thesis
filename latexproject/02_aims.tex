\chapter{Aims and scope of this thesis}
\label{aims}

Co-crystals of protein-GAG complexes are difficult to obtain, since GAGs feature
a high degree of conformational flexibility and biologically relevant
protein-GAG complexes are not always characterized by a high binding affinity.
NMR assignments of longer oligosaccharides are complicated or impossible to
obtain due to their repetitive nature. Furthermore, chemical synthesis of
oligosaccharides is very difficult. In short, the systematic experimental
investigation of protein-GAG systems is highly challenging and resource
intensive. In the past, important advanced have been made in the field of the
\textit{computational} investigation of these systems, with pioneering work
performed especially by the groups of Mancera and Imberty
\cite{imberty_perez_protgag_comp_book_2006, imberty_gag_prot_carbres_2007,
gandhi_structure_2008}. One of the reasons why \textit{in silico} investigations
of protein-GAG systems can be of particularly large value is the conceptually
straight-forward representation of arbitrary GAG molecules in simulation
systems. Also, as known from other fields of research, the effect of amino acid
mutations on a certain receptor-ligand system, for instance, can often reliably
be described using modern \textit{in silico} simulation approaches. An
interesting work highlighting the importance of combining experimental and
theoretical approaches in order to obtain a clearer understanding of the
molecular recognition properties of protein-GAG systems has been published by
the group of Prof. Huster at the Universität Leipzig in collaboration with our
group, about the molecular mechanism of the IL-8-GAG interaction
\cite{pichert_characterization_2012}.

Motivated by the encouraging groundwork done in the field of \textit{i)}
experimental IL-10-GAG system investigation and \textit{ii)} the \textit{in
silico} investigation of protein-GAG systems in general, the aim of this project
was to unravel atomic details of IL-10-GAG interaction with theoretical and
computational means. Furthermore, since the \textit{in silico} methods for
protein-GAG investigation are still in their infancy, this thesis project
included from the beginning, if required, the development and improvement of
methodological approaches. From the beginning of the project onwards, it was the
plan to at some point integrate \textit{in silico}-based predictions with
experimental results from collaborators at Prof. Huster's laboratory in Leipzig.
The ultimate goal was to be able to provide insights into the molecular
mechanisms determining IL-10-GAG interaction. Furthermore, methodology developed
during this project should be applicable to protein-GAG systems in general,
rendering it valuable for a large field of research.

Specifically, one of the first questions to be answered in this project was
where on the IL-10 homodimeric surface GAGs could potentially bind. Based on
such information, a subsequent fine-grained investigation should identify all
special determinants of IL-10-GAG interaction with atomic detail. For instance,
two important questions are whether there are certain key amino acid residues in
IL-10 that are especially relevant for GAG binding or whether there is binding
specificity of IL-10 for certain GAGs. The ultimate goal of this project was to
clarify the molecular mechanism of IL-10-GAG interaction and its possible
implications for IL-10's biological function. A resulting vision would then be
to be able to gain control over IL-10 function within artificial extracellular
matrices for improved tissue regeneration.

This research is in line with studies on cytokine-GAG interaction performed
earlier in other groups, such as the experimental investigation of IL-2-GAG
interaction, by which strong evidence was found that binding of HP and HS to
IL-2 happens, but does not interfere with IL-2/IL-2 receptor interaction
\cite{il2_gags_rider_1997}. In further studies, the authors concluded that
the IL-2-GAG interaction may be a mechanism for retaining the cytokine in an
active form close to its site of secretion in the tissue, an thus supporting the
paracrine role of IL-2 \cite{il2_gags_1998}. Also in case of IL-5 it was found
that it binds HP and HS and that it may modulate its biological function
\cite{il5_gags_coombe_1998}. The authors speculate that manufacturing a
GAG that binds IL-5 and prevents its localization in biologically responsive
tissue could enable the development of a new approach to controlling certain
diseases. Furthermore, IL-7 was found to bind to HP and HS, and the authors
speculated that HP may act as a carrier for IL-7, blocking its interaction with
target cells and protecting it from degradation during transit
\cite{il7_gags_1995}.

All of these cytokine-GAG investigations and their diverse --- and often vague
--- conclusions point out the importance of understanding the underlying
molecular mechanims, and principally serve as a template for what can and
should be found out about a certain cytokine-GAG system such as the
IL-10-GAG system is one.