\chapter{Introduction}


\section{Glycosaminoglycans (GAGs)}

Glycosaminoglycans (GAGs) play a critical role in many biological processes.
Their multifarious biological activity arises from their ability to interact
with and regulate a large number of proteins \cite{handel_2005}.


\section{Interleukin-10 (IL-10), and its relation to GAGs}

We are interested in GAG interaction with the cytokine interleukin-10 (IL-10,
reviewed in \cite{moore_2001}), which is generally considered to exert an
immunosuppressive function. From \textit{in vitro} experiments, IL-10 is known
to bind GAGs and there is evidence that GAGs may modulate its biological
function \cite{salek_ardakani_2000}. So far however, no structural detail about
IL-10-GAG interaction is known.


\section{Aim and scope of this project}

The aim of this project is to unravel atomic details of IL-10-GAG interaction
with theoretical and computational means. If required, methodological approaches
are to be developed. Integration of \textit{in silico}-based predictions with
experimental results from collaborators will hopefully provide insights into the
mechanisms determining IL-10-GAG interaction. Methodology developed during this
project is applicable to protein-GAG systems in general, rendering it valuable
for a large field of research.



\hl{cite interleukin-2 -heparin interaction (mulloy, rider) !}

\lipsum[1-5]