% Copyright 2014 Jan-Philip Gehrcke

% http://www.howtotex.com/packages/the-nag-package-warns-you-for-incorrect-latex-usage/
\RequirePackage[l2tabu, orthodox]{nag}
\documentclass[a4paper,12pt,onecolumn,twoside,openright]{memoir}

% Use Linux Libertine font. A quote from Michael Sharpe: "The free font Linux
% Libertine is one particular target — it is of nearly the same x-height as
% Computer Modern, but, not being a modern font, does not have a high contrast
% ratio, and so appears denser than Computer Modern but not as much so as Times.
% It is meant as a replacement for Times, but differs from it in many
% characteristics, more similar to MinionPro than Times, and provides a better
% range of variants than Times — three weights (regular, semi-bold and bold)
% rather than just two, and has expert features in all weights: old-style
% figures, more extensive and more interesting ligatures, and small caps. In my
% opinion, material typeset in Linux Libertine looks better than the
% corresponding material typeset in Times. This seems especially true on the
% screen."

% Math font from newtx package, fits the Libertine text font very well.
% Must be loaded before fontspec, and fontspec must be loaded with the
% no-math option, followed by \usepackage{libertine} or
% \setmainfont{Linux Libertine O}, see
% http://tex.stackexchange.com/q/97299/11816
\usepackage[libertine]{newtxmath}
\usepackage[no-math]{fontspec}
\usepackage{libertine}

% Properly typeset units, and numbers with units.
\usepackage{siunitx}

% This is likely already loaded by some other package.
\usepackage{amsmath}

% Bold math symbols.
\usepackage{bm}

% Improve typography. Must be loaded after any font.
\usepackage{microtype}

% Dummy paragraphs.
\usepackage{lipsum}

\usepackage{graphicx}

% Highlight text via \hl{}.
\usepackage{soul}

% Temporary margin change.
%\usepackage[strict]{changepage}

% From my master thesis, will I use those?
\usepackage{booktabs}

% For SRED table in DMD chapter. TODO: Convert to not require that package.
\usepackage{threeparttable}

% insert other pdf documents
\usepackage{pdfpages}

\usepackage{url}

% captions (for figures etc)
%\DisemulatePackage{ccaption}
\usepackage[margin=5pt,font=small,labelfont=bf]{caption}

% Configure footnotes: counter reset every page etc.
\usepackage[perpage,hang,splitrule]{footmisc}

\usepackage[english]{babel}
\usepackage[babel]{csquotes}
\usepackage{color}
\usepackage{fancyvrb}
\usepackage{float}

% ftp://ftp.tu-chemnitz.de/pub/tex/macros/latex/contrib/nomencl/nomencl.pdf
% http://tex.stackexchange.com/a/100377
\usepackage[intoc]{nomencl}

% http://stackoverflow.com/a/8649145/145400 LaTeX won't hyphenate words with
% dashes in them. There's a standard package that addresses that very problem,
% called extdash. This defines new hyphen and dash commands that do not disrupt
% hyphenation, and which can allow or prevent line breaks at the hyphen/dash. I
% prefer to use it with the shortcuts option, so I can use, e.g., \-/ rather
% than \Hyphdash
\usepackage[shortcuts]{extdash}

% Intelligent space after "commands".
% http://tex.stackexchange.com/a/17873/11816
\usepackage{xspace}

% Add bibliography to ToC.
% http://en.wikibooks.org/wiki/LaTeX/Bibliography_Management
%\usepackage[nottoc]{tocbibind}

% Hyperref provides PDF links etc. Should be included as late as possible.
\usepackage{hyperref}

% See: Which packages should be loaded after hyperref instead of before?
% http://tex.stackexchange.com/q/1863

% This package introduces the \cref command. When using this command to make
% cross-references, instead of \ref or \eqref, a word is placed in front of the
% reference according to the type of reference: fig. for figures, eq. for
% equations.
% noabbrev option for non-abbreviated names.
\usepackage[noabbrev]{cleveref}


\definecolor{sepia}{cmyk}{0, 0.83, 1, 0.70}
\hypersetup{
    pdfauthor={Jan-Philip Gehrcke},
    pdftitle={Investigation of the interleukin-10-GAG interaction using molecular
        simulation methods},
    pdfsubject={PhD thesis, computational biophysics},
    pdfkeywords={one, two},
    colorlinks=true,
    linkcolor=sepia,
    citecolor=sepia,
    urlcolor=sepia
}

%\renewcommand{\unspace}{\relax}
% For styles, see
% ftp://ftp.rrzn.uni-hannover.de/pub/mirror/tex-archive/macros/latex/contrib/biblatex-contrib/
\usepackage[
    backend=biber,
    citestyle=numeric,
    %bibstyle=numeric,
    bibstyle=ieee,
    %citestyle=authoryear,
    %bibstyle=authoryear,
    hyperref=true,
    backref=true,
    sorting=none,
    indexing=true,
    sortcites=true,
    url=false,
    isbn=true,
    eprint=false,
    doi=true,
    ]{biblatex}

\addbibresource{literature.bib}
%\bibliography{literature}
%{\small\bibliography{literature.bib}}


% Footnote layout config
%\addtolength{\footskip}{0.5cm}
%\setlength{\footnotemargin}{0.2cm}
%\setlength{\footnotesep}{0.4cm}
%\makeatletter
%\let\splitfootnoterule=\pagefootnoterule
%\makeatother


% *************** Memoir page layout options ***************
\settypeblocksize{*}{\lxvchars}{1.618}
% First field: space between top page border and text.
\setulmargins{5cm}{*}{*}

% First field is "inner margin", defining the horizontal position of text block.
\setlrmargins{5cm}{*}{*}
%\setlrmargins{*}{*}{1.618}

%\setheadfoot{headheight}{footskip}
% footskip: text bottom to footer bottom
% headheight: height of header block
\setheadfoot{2\onelineskip}{2\onelineskip}

% Second field: space between text and header.
%\setheaderspaces{*}{2.5cm}{*}
\setheaderspaces{*}{4\onelineskip}{*}
\checkandfixthelayout


% *************** Chapter and section and head style ***************
%\chapterstyle{southall}
%\chapterstyle{veelo}
%\chapterstyle{bianchi}

% \setsecheadstyle{\Large\sffamily\bfseries}
% \setsubsecheadstyle{\large\sffamily\bfseries}
% \setsubsubsecheadstyle{\normalfont\sffamily\bfseries}
% \setparaheadstyle{\normalfont\sffamily}

% *************** Table of contents style ***************
%\settocdepth{section}
\setsecnumdepth{subsection}
\maxsecnumdepth{subsection}
\maxtocdepth{subsection}


% % ********** Commands for epigraphs **********
% \setlength{\epigraphwidth}{0.57\textwidth}
% \setlength{\epigraphrule}{0pt}
% \setlength{\beforeepigraphskip}{1\baselineskip}
% \setlength{\afterepigraphskip}{2\baselineskip}

% \newcommand{\epitext}{\sffamily\itshape}
% \newcommand{\epiauthor}{\sffamily\scshape ---~}
% \newcommand{\epititle}{\sffamily\itshape}
% \newcommand{\epidate}{\sffamily\scshape}
% \newcommand{\episkip}{\medskip}

% % temporal margin change
% % usage: \begin{changemargin}{-1cm}{-1cm}
% \newenvironment{changemargin}[2]{%
%  \begin{list}{}{%
%   \setlength{\topsep}{0pt}%
%   \setlength{\leftmargin}{#1}%
%   \setlength{\rightmargin}{#2}%
%   \setlength{\listparindent}{\parindent}%
%   \setlength{\itemindent}{\parindent}%
%   \setlength{\parsep}{\parskip}%
%  }%
% \item[]}{\end{list}}

% special hyphenation
% \hyphenation{na-no-par-ti-cles pe-rio-di-ci-ty T-pe-rio-di-ci-ty Lange-vin}
% \setlength{\emergencystretch}{1em}

% URL font style
% http://tex.stackexchange.com/a/109750/11816
\urlstyle{same}

% Define Angstrom command.
%\newcommand{\angstrom}{\textup{\AA}\xspace}
% http://tex.stackexchange.com/a/24282/11816
\sisetup{
    per-mode=symbol,
    text-angstrom={Å},
    math-angstrom={\text{Å}},
    separate-uncertainty,
    }
\DeclareSIUnit\calory{cal}

% Set smaller font size in bibliography.
\renewcommand*{\bibfont}{\small}

% http://tex.stackexchange.com/a/19678
\newcommand{\specialcell}[2][c]{%
  \begin{tabular}[#1]{@{}c@{}}#2\end{tabular}}

% A new float type 'listing' for presenting
% sourcecode.
\newfloat{listing}{tbph}{lop}[chapter]
\floatname{listing}{Listing}

% Date/Signature field
% http://tex.stackexchange.com/a/35943
\newcommand{\namesigdate}[2][5cm]{%
  \begin{tabular}{@{}p{#1}@{}}
    #2 \\[2\normalbaselineskip] \hrule \\[0pt]
    {\small \textit{Unterschrift}} \\[2\normalbaselineskip] \hrule \\[0pt]
    {\small \textit{Datum}}
  \end{tabular}
}


% ---- Definitions for Python code highlight BEGIN ----
% $ pygmentize -f latex -S default
\makeatletter
\def\PY@reset{\let\PY@it=\relax \let\PY@bf=\relax%
    \let\PY@ul=\relax \let\PY@tc=\relax%
    \let\PY@bc=\relax \let\PY@ff=\relax}
\def\PY@tok#1{\csname PY@tok@#1\endcsname}
\def\PY@toks#1+{\ifx\relax#1\empty\else%
    \PY@tok{#1}\expandafter\PY@toks\fi}
\def\PY@do#1{\PY@bc{\PY@tc{\PY@ul{%
    \PY@it{\PY@bf{\PY@ff{#1}}}}}}}
\def\PY#1#2{\PY@reset\PY@toks#1+\relax+\PY@do{#2}}

\expandafter\def\csname PY@tok@gd\endcsname{\def\PY@tc##1{\textcolor[rgb]{0.63,0.00,0.00}{##1}}}
\expandafter\def\csname PY@tok@gu\endcsname{\let\PY@bf=\textbf\def\PY@tc##1{\textcolor[rgb]{0.50,0.00,0.50}{##1}}}
\expandafter\def\csname PY@tok@gt\endcsname{\def\PY@tc##1{\textcolor[rgb]{0.00,0.27,0.87}{##1}}}
\expandafter\def\csname PY@tok@gs\endcsname{\let\PY@bf=\textbf}
\expandafter\def\csname PY@tok@gr\endcsname{\def\PY@tc##1{\textcolor[rgb]{1.00,0.00,0.00}{##1}}}
\expandafter\def\csname PY@tok@cm\endcsname{\let\PY@it=\textit\def\PY@tc##1{\textcolor[rgb]{0.25,0.50,0.50}{##1}}}
\expandafter\def\csname PY@tok@vg\endcsname{\def\PY@tc##1{\textcolor[rgb]{0.10,0.09,0.49}{##1}}}
\expandafter\def\csname PY@tok@m\endcsname{\def\PY@tc##1{\textcolor[rgb]{0.40,0.40,0.40}{##1}}}
\expandafter\def\csname PY@tok@mh\endcsname{\def\PY@tc##1{\textcolor[rgb]{0.40,0.40,0.40}{##1}}}
\expandafter\def\csname PY@tok@go\endcsname{\def\PY@tc##1{\textcolor[rgb]{0.53,0.53,0.53}{##1}}}
\expandafter\def\csname PY@tok@ge\endcsname{\let\PY@it=\textit}
\expandafter\def\csname PY@tok@vc\endcsname{\def\PY@tc##1{\textcolor[rgb]{0.10,0.09,0.49}{##1}}}
\expandafter\def\csname PY@tok@il\endcsname{\def\PY@tc##1{\textcolor[rgb]{0.40,0.40,0.40}{##1}}}
\expandafter\def\csname PY@tok@cs\endcsname{\let\PY@it=\textit\def\PY@tc##1{\textcolor[rgb]{0.25,0.50,0.50}{##1}}}
\expandafter\def\csname PY@tok@cp\endcsname{\def\PY@tc##1{\textcolor[rgb]{0.74,0.48,0.00}{##1}}}
\expandafter\def\csname PY@tok@gi\endcsname{\def\PY@tc##1{\textcolor[rgb]{0.00,0.63,0.00}{##1}}}
\expandafter\def\csname PY@tok@gh\endcsname{\let\PY@bf=\textbf\def\PY@tc##1{\textcolor[rgb]{0.00,0.00,0.50}{##1}}}
\expandafter\def\csname PY@tok@ni\endcsname{\let\PY@bf=\textbf\def\PY@tc##1{\textcolor[rgb]{0.60,0.60,0.60}{##1}}}
\expandafter\def\csname PY@tok@nl\endcsname{\def\PY@tc##1{\textcolor[rgb]{0.63,0.63,0.00}{##1}}}
\expandafter\def\csname PY@tok@nn\endcsname{\let\PY@bf=\textbf\def\PY@tc##1{\textcolor[rgb]{0.00,0.00,1.00}{##1}}}
\expandafter\def\csname PY@tok@no\endcsname{\def\PY@tc##1{\textcolor[rgb]{0.53,0.00,0.00}{##1}}}
\expandafter\def\csname PY@tok@na\endcsname{\def\PY@tc##1{\textcolor[rgb]{0.49,0.56,0.16}{##1}}}
\expandafter\def\csname PY@tok@nb\endcsname{\def\PY@tc##1{\textcolor[rgb]{0.00,0.50,0.00}{##1}}}
\expandafter\def\csname PY@tok@nc\endcsname{\let\PY@bf=\textbf\def\PY@tc##1{\textcolor[rgb]{0.00,0.00,1.00}{##1}}}
\expandafter\def\csname PY@tok@nd\endcsname{\def\PY@tc##1{\textcolor[rgb]{0.67,0.13,1.00}{##1}}}
\expandafter\def\csname PY@tok@ne\endcsname{\let\PY@bf=\textbf\def\PY@tc##1{\textcolor[rgb]{0.82,0.25,0.23}{##1}}}
\expandafter\def\csname PY@tok@nf\endcsname{\def\PY@tc##1{\textcolor[rgb]{0.00,0.00,1.00}{##1}}}
\expandafter\def\csname PY@tok@si\endcsname{\let\PY@bf=\textbf\def\PY@tc##1{\textcolor[rgb]{0.73,0.40,0.53}{##1}}}
\expandafter\def\csname PY@tok@s2\endcsname{\def\PY@tc##1{\textcolor[rgb]{0.73,0.13,0.13}{##1}}}
\expandafter\def\csname PY@tok@vi\endcsname{\def\PY@tc##1{\textcolor[rgb]{0.10,0.09,0.49}{##1}}}
\expandafter\def\csname PY@tok@nt\endcsname{\let\PY@bf=\textbf\def\PY@tc##1{\textcolor[rgb]{0.00,0.50,0.00}{##1}}}
\expandafter\def\csname PY@tok@nv\endcsname{\def\PY@tc##1{\textcolor[rgb]{0.10,0.09,0.49}{##1}}}
\expandafter\def\csname PY@tok@s1\endcsname{\def\PY@tc##1{\textcolor[rgb]{0.73,0.13,0.13}{##1}}}
\expandafter\def\csname PY@tok@sh\endcsname{\def\PY@tc##1{\textcolor[rgb]{0.73,0.13,0.13}{##1}}}
\expandafter\def\csname PY@tok@sc\endcsname{\def\PY@tc##1{\textcolor[rgb]{0.73,0.13,0.13}{##1}}}
\expandafter\def\csname PY@tok@sx\endcsname{\def\PY@tc##1{\textcolor[rgb]{0.00,0.50,0.00}{##1}}}
\expandafter\def\csname PY@tok@bp\endcsname{\def\PY@tc##1{\textcolor[rgb]{0.00,0.50,0.00}{##1}}}
\expandafter\def\csname PY@tok@c1\endcsname{\let\PY@it=\textit\def\PY@tc##1{\textcolor[rgb]{0.25,0.50,0.50}{##1}}}
\expandafter\def\csname PY@tok@kc\endcsname{\let\PY@bf=\textbf\def\PY@tc##1{\textcolor[rgb]{0.00,0.50,0.00}{##1}}}
\expandafter\def\csname PY@tok@c\endcsname{\let\PY@it=\textit\def\PY@tc##1{\textcolor[rgb]{0.25,0.50,0.50}{##1}}}
\expandafter\def\csname PY@tok@mf\endcsname{\def\PY@tc##1{\textcolor[rgb]{0.40,0.40,0.40}{##1}}}
\expandafter\def\csname PY@tok@err\endcsname{\def\PY@bc##1{\setlength{\fboxsep}{0pt}\fcolorbox[rgb]{1.00,0.00,0.00}{1,1,1}{\strut ##1}}}
\expandafter\def\csname PY@tok@kd\endcsname{\let\PY@bf=\textbf\def\PY@tc##1{\textcolor[rgb]{0.00,0.50,0.00}{##1}}}
\expandafter\def\csname PY@tok@ss\endcsname{\def\PY@tc##1{\textcolor[rgb]{0.10,0.09,0.49}{##1}}}
\expandafter\def\csname PY@tok@sr\endcsname{\def\PY@tc##1{\textcolor[rgb]{0.73,0.40,0.53}{##1}}}
\expandafter\def\csname PY@tok@mo\endcsname{\def\PY@tc##1{\textcolor[rgb]{0.40,0.40,0.40}{##1}}}
\expandafter\def\csname PY@tok@kn\endcsname{\let\PY@bf=\textbf\def\PY@tc##1{\textcolor[rgb]{0.00,0.50,0.00}{##1}}}
\expandafter\def\csname PY@tok@mi\endcsname{\def\PY@tc##1{\textcolor[rgb]{0.40,0.40,0.40}{##1}}}
\expandafter\def\csname PY@tok@gp\endcsname{\let\PY@bf=\textbf\def\PY@tc##1{\textcolor[rgb]{0.00,0.00,0.50}{##1}}}
\expandafter\def\csname PY@tok@o\endcsname{\def\PY@tc##1{\textcolor[rgb]{0.40,0.40,0.40}{##1}}}
\expandafter\def\csname PY@tok@kr\endcsname{\let\PY@bf=\textbf\def\PY@tc##1{\textcolor[rgb]{0.00,0.50,0.00}{##1}}}
\expandafter\def\csname PY@tok@s\endcsname{\def\PY@tc##1{\textcolor[rgb]{0.73,0.13,0.13}{##1}}}
\expandafter\def\csname PY@tok@kp\endcsname{\def\PY@tc##1{\textcolor[rgb]{0.00,0.50,0.00}{##1}}}
\expandafter\def\csname PY@tok@w\endcsname{\def\PY@tc##1{\textcolor[rgb]{0.73,0.73,0.73}{##1}}}
\expandafter\def\csname PY@tok@kt\endcsname{\def\PY@tc##1{\textcolor[rgb]{0.69,0.00,0.25}{##1}}}
\expandafter\def\csname PY@tok@ow\endcsname{\let\PY@bf=\textbf\def\PY@tc##1{\textcolor[rgb]{0.67,0.13,1.00}{##1}}}
\expandafter\def\csname PY@tok@sb\endcsname{\def\PY@tc##1{\textcolor[rgb]{0.73,0.13,0.13}{##1}}}
\expandafter\def\csname PY@tok@k\endcsname{\let\PY@bf=\textbf\def\PY@tc##1{\textcolor[rgb]{0.00,0.50,0.00}{##1}}}
\expandafter\def\csname PY@tok@se\endcsname{\let\PY@bf=\textbf\def\PY@tc##1{\textcolor[rgb]{0.73,0.40,0.13}{##1}}}
\expandafter\def\csname PY@tok@sd\endcsname{\let\PY@it=\textit\def\PY@tc##1{\textcolor[rgb]{0.73,0.13,0.13}{##1}}}

\def\PYZbs{\char`\\}
\def\PYZus{\char`\_}
\def\PYZob{\char`\{}
\def\PYZcb{\char`\}}
\def\PYZca{\char`\^}
\def\PYZam{\char`\&}
\def\PYZlt{\char`\<}
\def\PYZgt{\char`\>}
\def\PYZsh{\char`\#}
\def\PYZpc{\char`\%}
\def\PYZdl{\char`\$}
\def\PYZhy{\char`\-}
\def\PYZsq{\char`\'}
\def\PYZdq{\char`\"}
\def\PYZti{\char`\~}
% for compatibility with earlier versions
\def\PYZat{@}
\def\PYZlb{[}
\def\PYZrb{]}
\makeatother
% ---- Definitions for Python code highlight END ----

% Declare hyphenation rules for certain words (these rules take global effect).
% The \hyphenation command allows you to give explicit instructions. Provided
% that the word will hyphenate at all (that is, it is not prevented from
% hyphenating by any of the other restrictions above), the command will override
% anything the hyphenation patterns might dictate. The command takes one or more
% hyphenated words as argument — \hyphenation{ana-lysis pot-able}; note that (as
% here, for analysis) you can use the command to overrule TeX’s choice of
% hyphenation
\hyphenation{
    he-xa-sac-cha-ride
    mo-no-sac-cha-ride
    di-sac-cha-ride
    te-tra-sac-cha-ride
    ace-tyl-glu-co-sa-mine
    }

% http://tex.stackexchange.com/a/100377
\makenomenclature