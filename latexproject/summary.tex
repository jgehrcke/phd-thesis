\chapter{Introduction and summary}
% Abgegeben werden muss, according to Promotionsordnung:
% 3. eine Dissertation in 4 Exemplaren, maschinenschriftlich und gebunden sowie
%15 Exemplare einer Kurzfassung (maximal drei Seiten);

% Das hier ist sowohl die interne als auch die externe Kurzfassung.


Glycosaminoglycans (GAGs) are a class of linear polysaccharides, built of
periodically occurring disaccharide units. GAGs are ubiquitous in the
extracellular matrix (ECM), where they exhibit multifarious biological
activities. This diversity arises from --- among others --- their ability to
interact with and regulate a large number of proteins, such as cytokines,
chemokines, and growth factors. As of the huge variety in their chemical
configuration, GAGs are further sub-classified into different types of GAGs
(heparin, for instance, is one of these sub-classes). Hence, GAGs are a
\textit{diverse} class of molecules, which surely contributes to the broadness
of their spectrum of biological functions. Through varying arrangements of
sulfate groups and different types of uronic acid monosaccharides, certain GAG
molecules can establish highly specific atomic contacts to certain proteins. One
of the best-studied examples is antithrombin-heparin, whose biologically
relevant interaction requires a specific pentasaccharide sequence
\cite{antithrombin-thrombin-heparin-2004}. It is valid to assume, however, that
various biologically relevant protein-GAG systems have still not been
discovered. In other cases, the fundamental protein-GAG interaction mechanism
has still not been clarified on the atomic level, as done for the
antithrombin-heparin system.

The cytokine interleukin-10 (IL-10) has been shown to bind GAGs. Additionally,
from in \textit{in vitro} experiments, there are indications that GAGs may
modulate IL-10's biological function \cite{salek_ardakani_2000}. So far,
however, no structural detail about IL-10-GAG interaction is known.
Function-wise, IL-10 is mainly considered to be immunosuppressive and
anti-inflammatory, but it in fact has the pleiotropic ability to positively
\textit{and} negatively influence the innate and adaptive immune system, i.e.\
it constitutes quite a complex regulation system on its own, and the role of
GAGs in this system is yet to be clarified. The fact that IL-10 and GAGs are
both are present in the ECM, together with the above-cited observations, give
rise to the assumption that IL-10-GAG interaction is of biological relevance,
and that understanding the impact of GAGs on IL-10 biology is important. A
promising approach for obtaining knowledge about the relevance of IL-10-GAG
interaction is the investigation of this system on the structural level, i.e.\
the identification and characterization of the molecular interaction mechanisms
that govern the IL-10-GAG system, on the atomic level. In this project, it was
the goal to obtain details of IL-10-GAG interaction with theoretical and
computational means, and with the help of experiments performed by
collaborators.

In the course of this project, three methods for the \textit{in silico}
investigation of protein-GAG systems in general were developed and applied to
the IL-10-GAG system. First of all, a systematic approach for the analysis of
the electrostatic potential of proteins was specified and validated

In case of IL-10, it lead to the prediction of a GAG binding site, as elaborated
in \cref{chapter:bspred}.

methodological approaches were developed,

, and



Wherever IL-10 is supposed to
regulate an immune response, for instance in the process of wound healing and
tissue regeneration, GAGs are


 IL-10's biological function is mainly considered to be
, but

 has pleiotropic effects in immunoregulation and
inflammation.




mainly








are still
not discovered.


Many of such interactions, however,




of GAGs features a
large diversity, in


have many essential roles in biology. They

 chemical
configuration


It is already known that , a certain chemical




enormously diverse and

chemical diversity







are play a critical role in many biological processes.
Their multifarious biological activity arises from their ability to interact
with and regulate a large number of proteins.

  To that end, including Dynamic Molecular Docking
(DMD), a novel molecular dynamics-based protocol for local docking of
protein-GAG systems and a spatial clustering approach optimized for the
interpretation of GAG docking solution ensembles. Evaluation of IL-10's
electrostatic potential led to the prediction of one major putative IL-10-GAG
binding region. Via in-depth analysis of this binding region with DMD and an
exhaustive evaluation of molecular dynamics data, IL-10's residue R107 was
identified to play a particularly important role in IL-10-GAG recognition.
Furthermore, interpretation of NMR spectroscopy data obtained by collaboration
partners led to the observation of GAG-internal structural changes upon binding
to IL-10. All this is valuable information for deepening the understanding of
the molecular mechanism governing IL-10-GAG interaction. If unraveled,
GAG-mediated bioactivity of IL-10 may be specifically exploited in artificial
extracellular matrices for improved wound healing, as is the goal of the DFG
Transregio 67.

\lipsum[1-4]