\chapter{Summary}
% Abgegeben werden muss, according to Promotionsordnung:
% 3. eine Dissertation in 4 Exemplaren, maschinenschriftlich und gebunden sowie
%15 Exemplare einer Kurzfassung (maximal drei Seiten);

% Das hier ist sowohl die interne als auch die externe Kurzfassung.


Glycosaminoglycans (GAGs) are a class of linear polysaccharides, built of
periodically occurring disaccharide units. GAGs are ubiquitous in the
extracellular matrix (ECM), where they exhibit multifarious biological
activities. This diversity arises from --- among others --- their ability to
interact with and regulate a large number of proteins, such as cytokines,
chemokines, and growth factors. As of the huge variety in their chemical
configuration, GAGs are further sub-classified into different types of GAGs,
such as heparin. Hence, GAGs comprise a \textit{diverse} class of molecules,
which surely contributes to the broadness of their spectrum of biological
functions. Through varying arrangements of sulfate groups and different types of
uronic acid monosaccharides, certain GAG molecules can establish highly specific
atomic contacts to certain proteins. One of the best-studied examples is
antithrombin-heparin, whose biologically relevant interaction requires a
specific pentasaccharide sequence \cite{antithrombin-thrombin-heparin-2004}. It
is valid to assume, however, that various biologically relevant protein-GAG
systems have still not been discovered or, in other cases, the molecular
interaction mechanism has still not been clarified on the atomic level, as
done for the antithrombin-heparin system.


From \textit{in vitro}
experiments, the anti-inflammatory cytokine interleukin-10 (IL-10) is known to
bind GAGs and there is evidence that GAGs may modulate its biological function.
So far however, no structural detail about IL-10-GAG interaction was known. In
this project, atomic details



are still
not discovered.


Many of such interactions, however,




of GAGs features a
large diversity, in


have many essential roles in biology. They

 chemical
configuration


It is already known that , a certain chemical




enormously diverse and

chemical diversity







are play a critical role in many biological processes.
Their multifarious biological activity arises from their ability to interact
with and regulate a large number of proteins.

 of IL-10-GAG interaction were unraveled with
theoretical and computational means. To that end, \textit{in silico}
methodological approaches were developed, including Dynamic Molecular Docking
(DMD), a novel molecular dynamics-based protocol for local docking of
protein-GAG systems and a spatial clustering approach optimized for the
interpretation of GAG docking solution ensembles. Evaluation of IL-10's
electrostatic potential led to the prediction of one major putative IL-10-GAG
binding region. Via in-depth analysis of this binding region with DMD and an
exhaustive evaluation of molecular dynamics data, IL-10's residue R107 was
identified to play a particularly important role in IL-10-GAG recognition.
Furthermore, interpretation of NMR spectroscopy data obtained by collaboration
partners led to the observation of GAG-internal structural changes upon binding
to IL-10. All this is valuable information for deepening the understanding of
the molecular mechanism governing IL-10-GAG interaction. If unraveled,
GAG-mediated bioactivity of IL-10 may be specifically exploited in artificial
extracellular matrices for improved wound healing, as is the goal of the DFG
Transregio 67.

\lipsum[1-4]