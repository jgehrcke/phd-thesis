
\vspace{-1.5cm}

\chapter{Summary}
% Abgegeben werden muss, according to Promotionsordnung:
% 3. eine Dissertation in 4 Exemplaren, maschinenschriftlich und gebunden sowie
%15 Exemplare einer Kurzfassung (maximal drei Seiten);

% Das hier ist sowohl die interne als auch die externe Kurzfassung.


Glycosaminoglycans (GAGs) are linear polysaccharides, built of periodically
occurring disaccharide units. GAGs are ubiquitous in the extracellular matrix
(ECM), where they exhibit multifarious biological activities. This diversity
arises from --- among others --- their ability to interact with and regulate a
large number of proteins, such as cytokines, chemokines, and growth factors. As
of the huge variety in their chemical configuration, GAGs are further
sub-classified into different types (heparin, for instance, is one of these
sub-classes). Hence, GAGs are a \textit{diverse} class of molecules, which
surely contributes to the broadness of their spectrum of biological functions.
Through varying arrangements of sulfate groups and different types of saccharide
units, individual GAG molecules can establish specific atomic contacts to
proteins. One of the best-studied examples is antithrombin-heparin, whose
biologically relevant interaction requires a specific pentasaccharide sequence
\cite{antithrombin-thrombin-heparin-2004}. It is valid to assume, however, that
various proteins are yet to be discovered whose biological function is in some
way affected by GAGs. In other cases, and this is true for the cytokine
interleukin-10 (IL-10), there are already experimental clues for a biologically
relevant protein-GAG interaction, but the details are still obscure and the
fundamental molecular interaction mechanism has still not been clarified.

IL-10 has been shown to bind GAGs \cite{salek_ardakani_2000}. So far, however,
no structural detail about IL-10-GAG interaction is known. Function-wise, IL-10
is mainly considered to be immunosuppressive and therefore anti-inflammatory,
but it in fact has the pleiotropic ability to influence the immune system in
both directions, i.e.\ it constitutes a complex regulation system on its own.
Therefore, the role of GAGs in this system is potentially substantial, but is
yet to be clarified. \textit{In vitro} experiments have yielded indications for
GAGs being able to modulate IL-10's biological function
\cite{salek_ardakani_2000}, and obviously IL-10 and GAGs are simultaneously
present in the ECM. This gives rise to the assumption that IL-10-GAG interaction
is of biological significance, and that understanding the impact of GAGs on
IL-10 biology is important --- from the basic research point of view, but also
for the development of therapies, potentially involving artificially designed
ECMs.

A promising approach for obtaining knowledge about the nature of IL-10-GAG
interaction is its investigation on the structural level, i.e.\ the
identification and characterization of the molecular interaction mechanisms that
govern the IL-10-GAG system. In this PhD project it was my goal to reveal
structural and molecular details about IL-10-GAG interaction with theoretical
and computational means, and with the help of experiments performed by
collaborators in the framework of the Collaborative Research Centre DFG
Transregio 67. For achieving this, I developed three methods for the
\textit{in silico} investigation of protein-GAG systems in general and
subsequently applied them to the IL-10-GAG system. Parts of that work have been
published in scientific journals, as outlined further below.

I proposed and validated a systematic approach for predicting GAG binding
regions on a given protein, based on the numerical simulation and analysis of
its Coulomb potential. One advantage of this method is its intrinsic ability to
provide clues about the reliability of the resulting prediction. Application of
this approach to IL-10 lead to the observation that its Coulomb attraction for
GAGs is significantly weaker than in case of exemplary protein-GAG systems (such
as FGF2-heparin). Still, a distinct IL-10-GAG binding region centered on the
residues R102, R104, R106, R107 of the human IL-10 sequence was identified. This
region can be assumed to play a major role in IL-10-GAG interaction, as
described in \cref{chapter:bspred}.

Molecular docking methods are used to generate binding mode predictions for a
given receptor-ligand system. In \cref{chapter:clustering}, I clarify the
importance of data clustering as an essential step for post-processing docking
results and present a novel clustering methodology optimized for GAG molecules.
It allows for a reproducible analysis, enabling systematic comparisons among
different docking studies. This clustering approach has become standard
procedure in our research group. It was applied in a variety of published
studies \cite{franz_cathepsin_2013, hintze_sergey_2014, Samsonov_rings_cr_2013,
SalbachHirsch20137653, vanderSmissen2013}, and was essential for studying
IL-10-GAG interaction, as described in \cref{chapter:dmdil10}.

Motivated by the shortcomings of classical docking approaches, especially with
respect to protein-GAG systems, I worked on the development of a molecular
dynamics-based docking method with less radical approximations than usually
applied in classical docking. The goal was to make the computational model
properly account for the special physical properties of GAGs, and to include the
effects of receptor flexibility and solvation, which are frequently ignored. The
novel methodology was named Dynamic Molecular Docking (DMD), and its concept was
published in the \textit{Journal of Chemical Information and Modeling}
\cite{dmd_samsonov_gehrcke_2014}, together with a validation study.

The subsequent application of DMD in a variety of studies required enormous
amounts of computational resources. For tackling this challenge, I established a
graphics processing unit-based high-performance computing environment in our
research group and developed a software framework for reliably performing DMD
studies on this hardware, as well as on other computing resources of the TU
Dresden. The investigation of the IL-10-GAG system via DMD was focused on the
IL-10-GAG binding region predicted earlier, and made heavy usage of the
optimized clustering approach named above. An important result of this endeavor
is that IL-10's amino acid residue R107 significantly stands out compared to all
other residues and supposedly plays a particularly important role in IL-10-GAG
recognition. The collaboration with the NMR laboratory of Prof. Daniel Huster at
the Universität Leipzig was fruitful: I post-processed nuclear Overhauser effect
data and obtained heparin structure models, which revealed that IL-10-heparin
interaction has a measurable impact on the backbone structure of the heparin
molecule. These results were published in \textit{Glycobiology}
\cite{kuenze_gehrcke_2014}. In \cref{together}, I propose two different
scenarios about how GAG-binding to IL-10 might affect its biological function,
based on the findings made in this thesis project.

In conclusion, a set of methods has been developed, all of which are generically
applicable for the investigation of protein-GAG systems. Regarding the IL-10-GAG
system, valuable structural insights for increasing the understanding about its
molecular mechanisms were derived. These observations pave the way towards
unraveling GAG-mediated bioactivity of IL-10, which may then be specifically
exploited, for instance in artificial ECMs for improved wound healing.
