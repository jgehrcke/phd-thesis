\chapter{Summary}
% Abgegeben werden muss, according to Promotionsordnung:
% 3. eine Dissertation in 4 Exemplaren, maschinenschriftlich und gebunden sowie
%15 Exemplare einer Kurzfassung (maximal drei Seiten);

% Das hier ist sowohl die interne als auch die externe Kurzfassung.

Glycosaminoglycans (GAGs) play a critical role in many biological processes.
Their multifarious biological activity arises from their ability to interact
with and regulate a large number of proteins. From \textit{in vitro}
experiments, the anti-inflammatory cytokine interleukin-10 (IL-10) is known to
bind GAGs and there is evidence that GAGs may modulate its biological function.
So far however, no structural detail about IL-10-GAG interaction was known. In
this project, atomic details of IL-10-GAG interaction were unraveled with
theoretical and computational means. To that end, \textit{in silico}
methodological approaches were developed, including Dynamic Molecular Docking
(DMD), a novel molecular dynamics-based protocol for local docking of
protein-GAG systems and a spatial clustering approach optimized for the
interpretation of GAG docking solution ensembles. Evaluation of IL-10's
electrostatic potential led to the prediction of one major putative IL-10-GAG
binding region. Via in-depth analysis of this binding region with DMD and an
exhaustive evaluation of molecular dynamics data, IL-10's residue R107 was
identified to play a particularly important role in IL-10-GAG recognition.
Furthermore, interpretation of NMR spectroscopy data obtained by collaboration
partners led to the observation of GAG-internal structural changes upon binding
to IL-10. All this is valuable information for deepening the understanding of
the molecular mechanism governing IL-10-GAG interaction. If unraveled,
GAG-mediated bioactivity of IL-10 may be specifically exploited in artificial
extracellular matrices for improved wound healing, as is the goal of the DFG
Transregio 67.

\lipsum[1-4]