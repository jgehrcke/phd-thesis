
\vspace{-0.5cm}

\chapter{Introduction and summary}
% Abgegeben werden muss, according to Promotionsordnung:
% 3. eine Dissertation in 4 Exemplaren, maschinenschriftlich und gebunden sowie
%15 Exemplare einer Kurzfassung (maximal drei Seiten);

% Das hier ist sowohl die interne als auch die externe Kurzfassung.


Glycosaminoglycans (GAGs) are a class of linear polysaccharides, built of
periodically occurring disaccharide units. GAGs are ubiquitous in the
extracellular matrix (ECM), where they exhibit multifarious biological
activities. This diversity arises from --- among others --- their ability to
interact with and regulate a large number of proteins, such as cytokines,
chemokines, and growth factors. As of the huge variety in their chemical
configuration, GAGs are further sub-classified into different types of GAGs
(heparin, for instance, is one of these sub-classes). Hence, GAGs are a
\textit{diverse} class of molecules, which surely contributes to the broadness
of their spectrum of biological functions. Through varying arrangements of
sulfate groups and different types of monosaccharides, individual GAG molecules
can establish highly specific atomic contacts to certain proteins. One of the
best-studied examples is antithrombin-heparin, whose biologically relevant
interaction requires a specific pentasaccharide sequence
\cite{antithrombin-thrombin-heparin-2004}. It is valid to assume, however, that
various proteins are yet to be discovered whose biological function is in some
way affected by GAGs. In other cases, and this is true for the cytokine
interleukin-10 (IL-10), there are already experimental clues for a biologically
relevant protein-GAG interaction, but the details are still obscure and the
fundamental molecular interaction mechanism has still not been clarified.

IL-10 has been shown to bind certain GAGs \cite{salek_ardakani_2000}. So far,
however, no structural detail about IL-10-GAG interaction is known.
Function-wise, IL-10 is mainly considered to be immunosuppressive and therefore
anti-inflammatory, but it in fact has the pleiotropic ability to influence the
immune system in both directions, i.e.\ it constitutes a complex regulation
system on its own. Therefore, the role of GAGs in this system is potentially
substantial, but is yet to be clarified. \textit{In vitro} experiments have
yielded indications for GAGs being able to modulate IL-10's biological function
\cite{salek_ardakani_2000}, and obviously IL-10 and GAGs are simultaneously
present in the ECM. This gives rise to the assumption that IL-10-GAG interaction
is of biological significance, and that understanding the impact of GAGs on
IL-10 biology is important --- from the basic research point of view, but also
for the development of therapeutic approaches.

A promising approach for obtaining knowledge about the nature of IL-10-GAG
interaction is the investigation of this system on the structural level, i.e.\
the identification and characterization of the molecular interaction mechanisms
that govern the IL-10-GAG system. In this PhD project it was my goal to reveal
structural and molecular details about IL-10-GAG interaction with theoretical
and computational means, and with the help of experiments performed by
collaborators. In the course of the project, three methods were developed for
the \textit{in silico} investigation of protein-GAG complexes and subsequently
applied to the IL-10-GAG system.

I proposed and validated a systematic approach for predicting GAG binding
regions on a given protein, based on the numerical simulation and analysis of
the Coulomb potential of the protein. One advantage of this method is its
intrinsic ability to provide clues about the reliability of the resulting
prediction. Application of this approach to IL-10 lead to the observation that
its Coulomb attraction for GAGs is significantly weaker than in case of
exemplary protein-GAG systems (such as FGF2-heparin). Still, a distinct
IL-10-GAG binding region centered on the residues R102, R104, R106, R107 of the
human IL-10 monomer was identified, and this region can be assumed to play a
major role in IL-10-GAG interaction, as described in \cref{chapter:bspred}.

Molecular docking methods are often used to generate a binding mode prediction
for a given receptor-ligand system. In \cref{chapter:clustering}, I clarify the
importance of data clustering as an essential step for post-processing an
ensemble of docking solutions. In the course of the PhD project, I developed a
clustering methodology optimized for GAG molecules. It allows for a reproducible
analysis, enabling systematic comparisons among different docking studies. This
clustering approach has become standard procedure in the Pisabarro research
group. It has so far been used in a variety of published studies (e.g.
\cite{franz_cathepsin_2013}), and also played an important role in the
investigation of the IL-10-GAG system, as described in \cref{chapter:dmdil10} of
this thesis.

Motivated by the shortcomings of classical docking approaches, especially with
respect to protein-GAG systems, we (a colleague of mine and I) developed a
molecular dynamics-based docking method with less radical approximations than
usually applied in classical docking. The goal was to make the computational
model properly account for the special physical properties of GAGs, and to
include the effects of flexibility and solvation. We named the method Dynamic
Molecular Docking (DMD), and successfully published its concept together with a
validation study in the \textit{Journal of Chemical Information and
Modeling} \cite{dmd_samsonov_gehrcke_2014}. The following application of DMD in a
variety of studies required enormous amounts of computational resources. For
tackling this challenge, I established a graphics processing unit-based high
performance computing (HPC) environment in our research group and developed a
software framework for reliably performing DMD studies on this hardware, as well
as on other HPC resources of the TU Dresden.

The subsequent investigation of the IL-10-GAG system via DMD was focused on the
IL-10-GAG binding region predicted earlier, and made heavy usage of the
optimized clustering approach named before. An important result of this endeavor
is that IL-10's amino acid residue R107 significantly stands out compared to all
other residues and supposedly plays a particularly important role in IL-10-GAG
recognition. Also the collaboration with the NMR laboratory of Prof. Daniel
Huster at the University Leipzig was fruitful: I post-processed nuclear
Overhauser effect data and obtained GAG structure models, which revealed that
IL-10-heparin interaction has a measurable impact on the backbone structure of
the heparin molecule. These results were published in \textit{Glycobiology}
\cite{kuenze_gehrcke_2014}. In terms of answering the question after the
biological impact of IL-10-GAG interaction, in \cref{together} I propose to
different scenarios about how GAG-binding to IL-10 might affect its biological
function.

In conclusion, a set of methods has been developed which is generically
applicable for the investigation of protein-GAG systems. Regarding the IL-10-GAG
system, valuable structural insights for increasing the understanding about is
molecular mechanims were derived. These observations pave the way towards
unraveling GAG-mediated bioactivity of IL-10, which may then be specifically
exploited, for instance in artificial ECMs for improved wound healing.
