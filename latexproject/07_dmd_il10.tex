\chapter{DMD applied to the IL-10-GAG system}

In order to exploit the predictive power of DMD for the IL-10-GAG system, we
have performed several DMD studies differing in DMD parameterization and
molecular system setup. Conducting these studies required an increment of the
computational resources available to us by orders of magnitudes. In the course
of performing these studies we continuously refined our software architecture
for controlling the corresponding high performance computing resources and
enriched our DMD data analysis framework. This chapter seeks to present
\textit{i)} the computational challenges, \textit{ii)} the methodological
details, and \textit{iii)} the major outcomes of this endeavor.


\section{Methods}

\subsection{DMD study design}

As a reminder, the term DMD study covers many DMD run repetitions followed by
data analysis. Hence, within a single DMD study the \textit{constants} are ---
among others --- the chemical configuration of the ligand molecule and the
geometrical DMD parametrization, i.e.\ the protein core atom and the focus
point. Clearly, for a systematic investigation of the IL-10-GAG system via DMD
we need to \textit{vary} exactly those. That is, that investigation requires the
conduction of \textit{many} DMD studies. By the time of designing the first
IL-10-GAG DMD studies, the conduction and analysis of a single study consumed
about two weeks, given the fact that no technical issues occurred. Hence, time
and computational resources clearly limited the scope of this endeavor. In these
lines, were required to get as much conclusive output of as few DMD studies as
possible.

\subsubsection{Factors to investigate}

The entities that are obviously interesting to vary among among the different
DMD studies in case of the IL-10-GAG system are:

\begin{itemize}

\item \textbf{The DMD protocol.} In the course of performing various DMD studies
we aimed for slight protocol changes for optimizing the computational efficiency
of DMD. We decided to potentially optimize parameters such as decreasing the tMD
simulation length as well as the LROM displacement length, both of which have
direct impact on the pulling process velocity. A the same time, we tried to
increase the number of independent DMD run repetitions within single DMD studies
as far as our computational resources allowed to, in order to obtain as much
sampling performance as possible.

\item \textbf{The geometrical DMD parameterization.} In \cref{chapter:bspred},
we have derived a region of interest for our DMD studies. For potentially
discovering a distinct and well-defined binding \textit{site} and for
investigating and finally eliminating the impact of the geometrical DMD
parameterization on any conclusion about the IL-10-GAG system, we decided to
vary this parameterization among the DMD studies. One should note however, that
in order to obtain reliable information about the impact of other variables,
this one here should not be varied too often, which is why we decided to try two
different conditions in the first round of DMD studies, and a final geometrical
parameterization in the second round of DMD studies.

\item \textbf{The representation of IL-10's N-terminus in the MD simulations}.
In IL-10's crystal structure with PDB 2ILK the first 5 residues of the
N-terminus are not resolved as of their flexibility. The N-terminus is close to
our main region of interest derived in \cref{chapter:bspred}. This is an
unfortunate situation, because if this terminus plays a role in GAG recognition,
it would not easily be possible for us to detect it -- modeling the behavior of
such flexible / disordered terminus is not possible with all certainty and
requires special sampling techniques on its own, not combinable with DMD.
Primarily, we take the flexibility and sequence as an argument that the
N-terminus is surely less important for specific GAG interaction than the region
we have identified. Since simulating this disordered part without special
treatment is pointless anyway, we have omitted these five residues from most of
our MD simulations. However, for having a comparison, and for obtaining an idea
about their possible impact, we decided to include them in at least one or two
test case DMD studies.

\item \textbf{The chemical configuration of GAG ligands.} Obviously, we had the
hope to identify characteristic differences among different GAG types and
lengths with respect to their interaction with our IL-10 region of interest.
Therefore, we aimed to perform DMD studies with constant conditions while
varying the chemical configuration of the GAG molecule. As of the limited amount
of computational resources, we had to decide which GAG types/lengths are most
interesting for our investigation and came up with the following selection:
\begin{itemize}
\item Heparin, a must-have in this list, as its interaction with IL-10 has been
experimentally confirmed previously \cite{salek_ardakani_2000}. We decided to
study heparin tetra- as well as hexasaccharides. One reason for this selection
is that in our DMD validation study, we have shown that DMD is capable of
properly sampling the internal degrees of freedom of GAG hexasaccharides.
Regarding longer GAGs, however, we would be uncertain again regarding the
sufficiency of the sampling and -- with that -- the reliability of the results.
Additional tetrasaccharide investigation is useful for observing potential
(in)consistenties in our observations depending on GAG length. Disaccharide

\item Hyaluronan
\item Chondroitin-4-sulfate
\item Chondroitin-6-sulfate
\end{itemize}



\end{itemize}

We decided to take an iterative approach. In a first round of DMD studies, we
aimed to gather experience and tried to estimate how the IL-10-GAG system
responds to various kinds of condition changes. Also, since the technical
conditions were unstable, we used this first round of studies for detecting
and resolving technical issues. In this first round of studies, we varied
the geometrical DMD parameterization


In a second round of studies,


        - iterative approach
            1st round, collect experience
            2nd round, start with most probable region
        - GAG ligand selection

First iteration


N= 200 or 300


Second iteration


\subsubsection{Automated DMD data analysis}

 list all analysis steps performed, i.e. conceptually clarifly the entire
 work flow here

 trajectories -> analysis -> merging -> clustering -> cluster stats, etc


\subsection{Computational framework}

\subsubsection{High performance computing resources}


 On state-of-the-art hardware, one
DMD study consumes about 10 years of CPU time. It can be conducted within two
weeks using a large resource pool in the supercomputing center of the TU Dresden (ZIH).
Independently, we have started establishing a group-internal compute cluster making
use of specialized GPU hardware. Using this, one DMD study can also be performed
in about two weeks. Recently, the ZIH has introduced GPU hardware. We have been
part of the testing period and started using their GPU resources for DMD production
runs.

For the purpose of being able to apply DMD on a scale as required by this study,
we have established a

        - Computing resources
            - Establishment of GPU infrastructure
                - Software: compilation of Amber 12 against ...
                - built up group-internal GPU cluster:
                    4 machines, this hardware, PBS schedular with Python wrapper
                - ZIH resources

\subsubsection{Software architecture}

        - Architecture:
            shell scripts, file system for study organization
            Python/numpy/scipy/matplotlib for analysis
            code on bitbucket



\section{Results and discussion}

\subsection{1st round of DMD studies}



So far, about 10 independent DMD studies involving IL-10 and different GAGs have
been performed. An important intermediate result of the corresponding data analysis
is that one amino acid residue, R107 (in the hIL-10 sequence, conserved in mouse), significantly
stands out compared to all other residues and supposedly plays a particularly
important role in IL-10-GAG recognition. This conclusion is based on different types
of data, including time-averaged hydrogen bonding analysis and single-residue energy
decomposition applied to MD data collected on the microsecond time scale

- Show picture of R107 specifically marked in the structure


R107 significantly stands out compared to all other residues
and supposedly plays a particularly important role in IL-10-
GAG recognition.
(conserved in mouse)

Observations in first round of studies:

\begin{itemize}

\item N=200 and N=300 for much clearer clustering and , reduction of target displacement length
to 25  Angstroms and pulling process simulation time to 3 ns instead of 4 without
observable caveats.

\item slight change in DMD geometry parameterization does not change outcome (different starting conditions, very similar outcome -- show figure on page 25 of 131030_3rd_tac )

\item Heparin length change yields same outcome

\item Complete inclusion of IL-10's N terminus does not qualitatively change hbond/SRED data (130530 groupmeeting pdf)

\item Again: DMD reproducible results!

\item MM-GBSA SRED yields consistent outcome with hydrogen bonding analysis,
whereas hbond analysis is much simpler


\end{itemize}

\subsection{2nd round of DMD studies}

\subsubsection{R107A mutation}

\hl{If there is not enough content here, do not make this a (sub)section}

Most important result: No other residue could take the role of R107,

