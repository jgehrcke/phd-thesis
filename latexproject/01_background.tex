\chapter{Background}
\label{background}

\section{Glycosaminoglycans (GAGs)}
\label{background:gags}

Carbohydrates are ubiquitous building blocks found in all forms of life. As
implicated by their name, they are made of carbon, hydrogen and oxygen. Despite
this rather small set of atom types, an enormous variety of carbohydrates exist.
The origin of major parts of this diversity is of combinatorial nature:
carbohydrates usually occur as polysaccharides made of many monosaccharides,
whereas different types of monosaccharides exist. They are distinguishable by
their chemical configuration, and usually there are two stereoisomers for each
of those, leading to a broad spectrum of different \enquote{sugars} whose
nomenclature and chemistry is described in detail in reference books
\cite{carbohydrate_chemistry_robyt_1998,carbohydrate_chemistry_royal_2000}.

Glycosaminoglycans (GAGs), reviewed in
\cite{essentials_glycobiology_gags_chapter_2009}, are a special class of
carbohydrates. They are unbranched linear saccharide chains, ideally comprised
of a repeating disaccharide unit, whereas each unit is made of two pyranose
monosaccharides (a pyranose is a six-membered ring consisting of five carbon
atoms and one oxygen atom):

\begin{itemize}
\item one is an amino sugar (or \enquote{hexosamine}), either a
D-N-acetylglucosamine (GlcN) or a D-N-acetylgalactosamine (GalN)
\item the other is a uronic saccharide (or \enquote{hexuronic acid}), either a
D-glucuronic acid (GlcA) or its C5-epimer L-iduronic acid (IdoA), or, in seldom
cases, a D-galactose (Gal).
\end{itemize}

A special feature of GAGs is that they are often sulfated at various positions,
leading to a number of possible sulfation patterns per repeating disaccharide
unit. An arbitrary combination of basic disaccharide unit and sulfation pattern
implies that the heterogeneity among GAGs is enormous. Still, physiologically
occurring GAGs are usually roughly categorized into a couple major GAG types.
Some of those are listed in \cref{tab:bg:gagtypes}, together with their common
disaccharide unit in proper IUPAC nomenclature, and their charge per
disaccharide.


\begin{table}
\scriptsize
\centering
\renewcommand{\arraystretch}{1.3}
\begin{tabular}{lll}
\midrule
GAG type & main disaccharide & charge/\si{elementarycharge} \\
\midrule
Heparin (HP) & L-IdoA2S-$\alpha$(1$\rightarrow$4)-D-GlcNS6S-$\alpha$(1$\rightarrow$4) & -4 \\
Chondroitin-4-sulfate (CS4) & D-GlcA-$\beta$(1$\rightarrow$3)-D-GalN4S-$\beta$(1$\rightarrow$4) & -2 \\
Hyaluronan (HA) & D-GlcA-$\beta$(1$\rightarrow$4)-D-GlcN-$\alpha$(1$\rightarrow$4) & -1 \\
\midrule
\end{tabular}
\caption{
Characterization of most populated clusters by single-run quantity cluster
statistics for two different DMD studies differing only in the GAG type. In one
study, a HP dp6 was used as ligand, in the other study a CS4 dp6 was used. The
HP study yielded a single prominent cluster, the CS4 study yielded two similar
clusters. $\epsilon$ is the spatial density of docking solutions according to
\cref{clustering_param_opt}. $m$ is the mobility of the ligand relative to the
receptor, as defined in \cref{dmd:dataanalysis}. Here, $\Delta G$ is the free
energy of binding estimate as provided by the MM-PBSA method
(\cref{methods:mmpbsa_mmgbsa}). The standard deviations shown are derived from
the cluster-internal variations.}
\label{tab:bg:gagtypes}
\end{table}



Furthermore, the naturally occurring sulfation of GAGs is tremendous, with up to
three sulfate groups per disaccharide unit. With the carboxyl group in the
hexuronic acid, each repeating disaccharide unit carries one negative charge at
physiological pH anyway. Considering sulfation, this charge may grow up to -4,
results in heparin being the biological macromolecule with the largest charge
density known \cite{capila_linhardt_hep_prot_2002}.


There are other variants with established names, such as Chondroitin-6-sulfate (CS6), which is the same as CS4, but sulfated at the C6 position of the galactosamine instead of in the fourth position.


Still,


 resulting in another differentiating feature apart from differences in disaccharide composition. The fact that this sulfation pattern can vary within any given type of GAG, implies that the heterogeneity of GAGs is enormous.



Additional consequence of these sulfate additions, is the highly negative charge of glycosaminoglycans. Together with the carboxyl group of the hexuronic acids, this


Most glycosaminoglycans can be found in large quantities in specific cell structures called proteoglycans. These large structures consist of a linear core protein with many glycosaminoglycans covalently linked to it. Each glycosaminoglycan is linked to the core protein via a Gal-Gal-Xyl sugar linker, where the xylose (Xyl) is attached to a serine residue of the core protein and the glycosaminoglycan is linked to the galactose (Gal) residue. Also proteoglycans are highly diverse; variations occur both in the type of core protein and in type, size and number of glycosaminoglycans.



In organisms, GAG chains usually
appear covalently bound to a core protein, comprising a proteoglycan complex.
To a large extent, the biological functions of proteoglycans depends on the
interaction of its GAG chain(s) with other proteins.



Considering the level of atomic detail aimed for ... to be observed... in this
thesis project, GAGs are modeled as short molecules. While it is likely
that their tremendous length and also their covalent linkage to proteoglycans
serve have an overall impact on the biological function of GAGs, it is a
valid and well-established approximation to not account for these facts in
molecular modeling studies that aim for resolving the molecular mechanism of
protein-GAG interaction in atomic detail... in a biological function and overall  treated as


are play a critical role in many biological processes.
Their multifarious biological activity arises from their ability to interact
with and regulate a large number of proteins.


Glycosaminoglycans (GAGs) play a critical role in many biological processes.
Their multifarious biological activity arises from their ability to interact
with and regulate a large number of proteins \cite{handel_2005}.

Introduce abbreviations (HP, HA, CS4, CS6, dpX, etc.)


degree of polymerization


The origin and geometry of various pyranose monosaccharide conformations is
comprehesibly classified and discussed in
\cite{classification_pyranose_conformers_1960}. The conformational nomenclature
used throughout this thesis follows IUPAC rules, which are well-described in
\cite{iupac_gag_conformations_1980}.


\section{A primer on IL-10 biology}

 IL-10's biological function is mainly considered to be
, but

 has pleiotropic effects in immunoregulation and
inflammation.


        in closing remarks, relate to extracellular matrix


\section{Interleukin-10 (IL-10), and its relation to GAGs}


 anti-inflammatory

- Relevance of IL-10 in immune regulation
- Bio-relevant IL-10-GAG interaction? Motivated by Salek ardakini.

We are interested in GAG interaction with the cytokine interleukin-10 (IL-10,
reviewed in \cite{moore_2001}), which is generally considered to exert an
immunosuppressive function. From \textit{in vitro} experiments, IL-10 is known
to bind GAGs and there is evidence that GAGs may modulate its biological
function \cite{salek_ardakani_2000}. So far however, no structural detail about
IL-10-GAG interaction is known.


Wherever IL-10 is supposed to
regulate an immune response, for instance in the process of wound healing and
tissue regeneration, GAGs are


\section{In-silico methods for investigating receptor-ligand interaction}

\subsection{A primer on docking methods}


    - Methods applied in literature: a mini review
        AutoDock3 stands out, just by experience, not by concept
    - Problems of different complexity: local vs. global docking.
    - (AutoDock 3 protein-GAG blind docking validation study,
        Method, results, discussion, conclusion)

\lipsum[1-5]

\subsection{A primer on molecular dynamics simulations}

\lipsum[1-5]

\subsubsection{End-point free energy methods (\enquote{MM-PB(GB)SA})}
% This label is required in DMD chapter.
\label{methods:mmpbsa_mmgbsa}


Cite this: \cite{schlick_innovationsdynamics_2012} (volume 2, chapter 12).

\lipsum[1-5]

\section{Structure of IL-10 and its receptors}

    - Structure description
        exists mainly as homodimer [
            biochemistry, 1998, 37, 16943-16951, zdanov 1995]


    - IL-10 and its receptors
        - what is known in literature (current state of knowlege)
            R1+IL-10 structure
            R2 structure
            ternary complex binding models
            what is required for signaling? literature overview
        - a critical review: monomer
            minimal unit required for signaling (monomer)
        - IL-10 + R1 + R2 structure model from IL20 ternary complex


\section{Aim and scope of this thesis}

- Vision: gaining control over IL-10 function in artificial extracellular matrices

- Motivation and goal: investigate IL-10-GAG system with computational
      methods, in collaboration with...

The aim of this project is to unravel atomic details of IL-10-GAG interaction
with theoretical and computational means. If required, methodological approaches
are to be developed. Integration of \textit{in silico}-based predictions with
experimental results from collaborators will hopefully provide insights into the
mechanisms determining IL-10-GAG interaction. Methodology developed during this
project is applicable to protein-GAG systems in general, rendering it valuable
for a large field of research.



\hl{cite interleukin-2 -heparin interaction (mulloy, rider) !}

\lipsum[1-5]





