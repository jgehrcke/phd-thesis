\chapter{Introduction}


\section{Glycosaminoglycans (GAGs)}
\label{intro:gags}


whereas each repeating unit
consists of an amino sugar and a uronic acid.

from wikipedia:
Pyranose is a collective term for carbohydrates that have a chemical structure that includes a six-membered ring consisting of five carbon atoms and one oxygen atom. The name derives from its similarity to the oxygen heterocycle pyran, but the pyranose ring does not have double bonds. A pyranose in which the anomeric OH at C(l) has been converted into an OR group is called a pyranoside.



In organisms, GAG chains usually
appear covalently bound to a core protein, comprising a proteoglycan complex.
To a large extent, the biological functions of proteoglycans depends on the
interaction of its GAG chain(s) with other proteins.



Considering the level of atomic detail aimed for ... to be observed... in this
thesis project, GAGs are modeled as short molecules. While it is likely
that their tremendous length and also their covalent linkage to proteoglycans
serve have an overall impact on the biological function of GAGs, it is a
valid and well-established approximation to not account for these facts in
molecular modeling studies that aim for resolving the molecular mechanism of
protein-GAG interaction in atomic detail... in a biological function and overall  treated as


are play a critical role in many biological processes.
Their multifarious biological activity arises from their ability to interact
with and regulate a large number of proteins.


Glycosaminoglycans (GAGs) play a critical role in many biological processes.
Their multifarious biological activity arises from their ability to interact
with and regulate a large number of proteins \cite{handel_2005}.

Introduce abbreviations (HP, HA, CS4, CS6, dpX, etc.)



The origin and geometry of various pyranose monosaccharide conformations is
comprehesibly classified and discussed in
\cite{classification_pyranose_conformers_1960}. The conformational nomenclature
used throughout this thesis follows IUPAC rules, which are well-described in
\cite{iupac_gag_conformations_1980}.




\section{Interleukin-10 (IL-10), and its relation to GAGs}


 anti-inflammatory

- Relevance of IL-10 in immune regulation
- Bio-relevant IL-10-GAG interaction? Motivated by Salek ardakini.

We are interested in GAG interaction with the cytokine interleukin-10 (IL-10,
reviewed in \cite{moore_2001}), which is generally considered to exert an
immunosuppressive function. From \textit{in vitro} experiments, IL-10 is known
to bind GAGs and there is evidence that GAGs may modulate its biological
function \cite{salek_ardakani_2000}. So far however, no structural detail about
IL-10-GAG interaction is known.


Wherever IL-10 is supposed to
regulate an immune response, for instance in the process of wound healing and
tissue regeneration, GAGs are







\section{Aim and scope of this project}

- Vision: gaining control over IL-10 function in artificial extracellular matrices

- Motivation and goal: investigate IL-10-GAG system with computational
      methods, in collaboration with...

The aim of this project is to unravel atomic details of IL-10-GAG interaction
with theoretical and computational means. If required, methodological approaches
are to be developed. Integration of \textit{in silico}-based predictions with
experimental results from collaborators will hopefully provide insights into the
mechanisms determining IL-10-GAG interaction. Methodology developed during this
project is applicable to protein-GAG systems in general, rendering it valuable
for a large field of research.



\hl{cite interleukin-2 -heparin interaction (mulloy, rider) !}

\lipsum[1-5]





