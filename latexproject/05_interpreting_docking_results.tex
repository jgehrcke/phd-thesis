\chapter{Interpreting docking results: docking solution ensemble clustering}

In this chapter we describe why the conduction of a meaningful docking study
requires proper creation and analysis of an ensemble of docking solutions and
explain why clustering is an essential step for making reasonable predictions
about where and how a ligand binds to its receptor. This topic is unfortunately
not well discussed in current molecular docking literature, and existing
technical implementations for the clustering of docking solutions often do not
fit the scientific question and therefore may produce misleading results. Here,
we present a method optimized for clustering docking solution ensembles of GAG
molecules.

\section{Relevance of docking solution clustering}
In the framework of this thesis the term \enquote{docking solution} shall be
understood as a set of ligand atom coordinates, i.e. the spatial arrangement of
ligand atoms in the reference coordinate system as defined by the receptor
molecule. As a result of the statistical nature of the docking problem and the
use of pseudo random numbers in implementations of common docking search
algorithms, independent runs of the same docking method are expected to produce
different docking solutions. That is why a molecular docking method should be
applied repetitively, resulting in an ensemble of docking solutions.

The spatial distribution of docking solutions in that ensemble is governed by a
certain previously unknown probability distribution, which itself is implicitly
determined by the molecular interaction model and search algorithm implemented
in the docking method. Given that we trust the molecular interaction model, the
ultimate goal of a docking study is to obtain the global maximum (and/or some
local maxima) of the named probability distribution. Concept-wise, this is
equivalent to searching the state with highest probability, i.e. lowest energy.
Obtaining local maxima requires complete sampling of that unknown probability
distribution. That is why a docking method must be repeated until convergence
in the spatial distribution of the docking solutions is achieved. Only then the
ensemble reproduces the previously unknown probability distribution and reflects
the molecular interaction model as implemented by the docking method.

If convergence is observed, the next step is to identify the local maxima in the
spatial probability distribution of docking solutions. This problem can be
reformulated in terms of density, i.e. as finding the densest agglomerations of
docking solutions in the ensemble. However, a simple evaluation of the
distribution of atoms or molecules per volume would not provide useful results,
because a ligand molecule is comprised of multiple different atoms and has
various asymmetries with respect to shape and property. Spoken freely, the task
is to find agglomerations of docking solutions that are highly \textit{similar}
to each other. Hence, a promising approach is to evaluate the density in terms
of the \textit{similarity} between any two given docking solutions in the
ensemble, subject to the condition that a meaningful molecular similarity
measure can be found. In an abstract sense, the task is to identify groups of
highly similar docking solutions and separate them from the bulk. This task fits
the general description of \textit{data clustering} as found in
\cite{tan_data_mining}:

\begin{adjustwidth}{1cm}{1cm}
\textit{
The goal is that the objects within a group be similar to one another and
different from the objects in other groups. The greater the similarity (or
homogeneity) within a group and the greater the difference between groups, the
better or more distinct the clustering.
}
\end{adjustwidth}

Any data clustering method has only two major ingredients:
\begin{itemize}
\item the \textit{distance metric}, quantifying the similarity between any two
given objects.
\item a \textit{clustering algorithm}, classifying the objects into groups,
based on their mutual similarity.
\end{itemize}

In conclusion, the most probable ligand molecule placement as predicted by a
certain docking method can be found by clustering the (converged) ensemble of
docking solutions, using a meaningful distance metric and a clustering algorithm
that fits the problem. It is important to appreciate that both ingredients must
be adjusted to the scientific question, otherwise the result of docking solution
clustering might be meaningless and/or incomparable to other docking studies.
Within the next sections, we introduce a distance metric optimized for GAG
molecules and describe a clustering algorithm appropriate for clustering
molecular structure ensembles in general.


\section{A meaningful distance metric for GAG molecules}
RMSatd


\section{Algorithm of choice: DBSCAN}

\subsection{Clustering method overview}

    - Clustering methods overview
        hierarchical clustering implemented in AutoDockTools
            badly implemented, due to $N^2$ dependence very long runtimes
        other methods, quick
        DBSCAN, explain why it makes sense for molecular structures
        cite cpptraj, which also started implementing it

\subsection{DBSCAN parameter optimization}

    - change criteria from eps/minp to N clusters / members
    - reproducibility!
    - comparability among studies
    - characterization of a cluster via eps


\section{Clustering method implementation}

\subsection{Efficient implementation of DBSCAN for numpy}

        - Custom implementation of the DBSCAN algorithm

\subsection{Efficient implementation of RMSatd for numpy}

        - Custom implementation of the DBSCAN algorithm
        - examples
            http://scikit-learn.org/stable/modules/clustering.html
            http://bit.ly/1pKNv7v


\subsection{Parameter optimization}

        - brute force plus local optimization

\subsection{Architecture of cluster-pdb-structures}

            command line tool, simple PDB parser files
            distance matrix calculation on multiple cores via Unix fork()

            input: structure ensemble, DBSCANopt parameters
            output: structure files, stat summary


            open source at bitbucket..


