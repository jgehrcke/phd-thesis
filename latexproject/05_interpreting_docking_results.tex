\chapter{Interpreting docking results: docking solution ensemble clustering}
\section{Relevance of docking solution clustering}

\hl{A few words on the fact that the importance of clustering usually is
neglected.}

In the framework of this thesis the term \enquote{docking solution} shall be
understood as a set of ligand atom coordinates, i.e. the spatial arrangement of
ligand atoms in the reference coordinate system as defined by the receptor
molecule. As a result of the statistical nature of the docking problem and the
use of pseudo random numbers in implementations of common docking search
algorithms, independent runs of the same docking method are expected to produce
different docking solutions. That is why a molecular docking method should be
applied repetitively, resulting in an ensemble of docking solutions.

The spatial distribution of docking solutions in that ensemble is governed by a
certain previously unknown probability distribution, which itself is determined
by the molecular interaction model and search algorithm implemented in the
docking method. Given that we trust the molecular interaction model, the
ultimate goal of a docking study is to obtain the maxima of the named
probability distribution. Concept-wise, this is equivalent to searching the
state with highest probability, i.e. lowest energy. When applying a docking
method, one should be interested in seeing convergence in the spatial
distribution of the docking solutions, because only then named probability
distribution has been sampled sufficiently and the ensemble reflects the
molecular interaction model as implemented by the docking method.

If convergence is observed, the next step is to identify the maxima in the
spatial probability distribution. This problem can be re-formulated in terms of
density, i.e. as finding the densest agglomerations of docking solutions in the
ensemble. However, a simple evaluation of the density distribution by atoms per
volume or molecules per volume would not be a useful approach, because a ligand
molecule usually is comprised of atoms of various type and property, has a
special shape, and sometimes a clear directionality. Freely spoken, we are
looking for a way to find agglomerations of docking solutions that are highly
\textit{similar} to each other. Hence, a promising approach is to evaluate the
density in terms of \textit{similarity} between any two given docking solutions
in the ensemble.




, and  whereas density is defined as





    -  mini review
        clustering in general
        clustering applied to docking ensemble: most probable state




docking solution ensemble governed by probability distribution





whereas density is defined as



\cite{tan_data_mining}


\section{A meaningful distance metric: RMSatd}

\section{Algorithm of choice: DBSCAN}

\subsection{Clustering method overview}

    - Clustering methods overview
        hierarchical clustering implemented in AutoDockTools
            badly implemented, due to $N^2$ dependence very long runtimes
        other methods, quick
        DBSCAN, explain why it makes sense for molecular structures
        cite cpptraj, which also started implementing it

\subsection{DBSCAN parameter optimization}

    - change criteria from eps/minp to N clusters / members
    - reproducibility!
    - comparability among studies
    - characterization of a cluster via eps


\section{Clustering method implementation}

\subsection{Efficient implementation of DBSCAN for numpy}

        - Custom implementation of the DBSCAN algorithm

\subsection{Efficient implementation of RMSatd for numpy}

        - Custom implementation of the DBSCAN algorithm
        - examples
            http://scikit-learn.org/stable/modules/clustering.html
            http://bit.ly/1pKNv7v


\subsection{Parameter optimization}

        - brute force plus local optimization

\subsection{Architecture of cluster-pdb-structures}

            command line tool, simple PDB parser files
            distance matrix calculation on multiple cores via Unix fork()

            input: structure ensemble, DBSCANopt parameters
            output: structure files, stat summary


            open source at bitbucket..


