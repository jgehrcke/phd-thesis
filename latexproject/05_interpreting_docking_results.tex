\chapter{Interpreting docking results: docking ensemble clustering}
\section{Relevance of clustering}


    -  mini review
        clustering in general
        clustering applied to docking ensemble: most probable state


Any docking method creates an ensemble of docking solutions.

docking solution ensemble governed by probability distribution

searching the state with highest probability

find the densest agglomeration of docking solutions in the ensemble

whereas density is defined as


\section{A meaningful distance metric: RMSatd}

\section{Algorithm of choice: DBSCAN}

\subsection{Clustering method overview}

    - Clustering methods overview
        hierarchical clustering implemented in AutoDockTools
            badly implemented, due to N^2 dependence very long runtimes
        other methods, quick
        DBSCAN, explain why it makes sense for molecular structures
        cite cpptraj, which also started implementing it

\subsection{DBSCAN parameter optimization}

    - change criteria from eps/minp to N clusters / members
    - reproducibility!
    - comparability among studies
    - characterization of a cluster via eps


\section{Clustering method implementation}

\subsection{Efficient implementation of DBSCAN for numpy}

        - Custom implementation of the DBSCAN algorithm

\subsection{Efficient implementation of RMSatd for numpy}

        - Custom implementation of the DBSCAN algorithm
        - examples
            http://scikit-learn.org/stable/modules/clustering.html
            http://bit.ly/1pKNv7v


\subsection{Parameter optimization}

        - brute force plus local optimization

\subsection{Architecture of cluster-pdb-structures}

            command line tool, simple PDB parser files
            distance matrix calculation on multiple cores via Unix fork()

            input: structure ensemble, DBSCANopt parameters
            output: structure files, stat summary


            open source at bitbucket..


